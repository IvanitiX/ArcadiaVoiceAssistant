\thispagestyle{empty}

\begin{center}
{\large\bfseries Asistente de voz modular \\ usando APIs libres }\\
\end{center}
\begin{center}
Iván Valero Rodríguez\\
\end{center}

%\vspace{0.7cm}

\vspace{0.5cm}
\noindent{\textbf{Palabras clave}: \textit{software libre, código abierto, asistentes virtuales, reconocimiento del habla, síntesis del habla, chatbots}
\vspace{0.7cm}

\noindent{\textbf{Resumen}\\
	Este Trabajo Fin de Grado trata de crear un Asistente Virtual por Voz teniendo como restricción el uso de herramientas y componentes de código abierto o libres, y que además permita el desarrollo de integraciones con otros componentes. Para ello, se hace una búsqueda de APIs que posean licencias de software libre y se analizan varias librerías para seleccionar las que tienen mejores resultados, siendo estos los que se usarán en conjunto con la codificación de un programa principal que permita las interacciones entre estos.
	\\ \\
	La metodología para la realización de este trabajo empieza con una mirada  exhaustiva al estado del arte para luego estudiar y comparar las herramientas y componentes necesarios en el diseño del software. Tras ello, y en base a ese diseño, se realizará el sistema anteriormente mencionado uniendo esos componentes y programando el flujo de interacción entre estos.
	\\ \\
	Además del desarrollo, se habla de cómo la sociedad puede verse afectada por este tipo de sistemas y de cómo el Software Libre puede ser beneficioso para los proyectos. En base a esos antecedentes, también se discute cómo el proyecto encajaría en este contexto tecnológico y social.
	
	
	

\cleardoublepage

\begin{center}
	{\large\bfseries Voice assistant using Open Source \\ and Free Software APIs}\\
\end{center}
\begin{center}
	Iván Valero Rodríguez\\
\end{center}
\vspace{0.5cm}
\noindent{\textbf{Keywords}: \textit{free software, open source, virtual assistants, speech recognition, speech synthesis, chatbots}
\vspace{0.7cm}

\noindent{\textbf{Abstract}\\
	
This Final Degree Project aims to create a Voice Virtual Assistant with the restriction of using open source or free tools and components, and which also allows the development of integrations with other components. To do so, research is carried out on APIs that have free software licences and some libraries are analysed to select the ones with the best results. These APIs are used together with a coding of a main program that allows the interactions between them.
\\ \\
The methodology for this project begins with an exhaustive look at the state of the art and after that, to study and compare the tools and components necessary for the design of the software. Based on this design, the above-mentioned system will be created by linking these components and programming the flow of interaction between them.
\\ \\
In addition to the development, it is discussed how society can be affected by such systems and how Free Software can be beneficial for projects. On the basis of this background, it is also discussed how the project would fit into this technological and social context.

	
\cleardoublepage

\thispagestyle{empty}

\noindent\rule[-1ex]{\textwidth}{2pt}\\[4.5ex]

Yo, \textbf{Iván Valero Rodríguez}, alumno de la titulación \textbf{Grado en Ingeniería Informática} de la \textbf{Escuela Técnica Superior de Ingenierías Informática y de Telecomunicación de la Universidad de Granada}, con DNI 459253405H, \textbf{autorizo} la ubicación de la siguiente copia de mi Trabajo Fin de Grado en la biblioteca del centro para que pueda ser consultada por las personas que lo deseen.
\vspace{0.5cm}

Y para que conste, expide y firma el presente informe en Granada a X de Julio de 2022.

\vspace{1cm}

\textbf{El estudiante: }

\vspace{5cm}

\noindent \textbf{D. Iván Valero Rodríguez}


\cleardoublepage

\thispagestyle{empty}

\noindent\rule[-1ex]{\textwidth}{2pt}\\[4.5ex]

D. \textbf{Pablo García Sánchez}, Profesor(a) del Departamento de Arquitectura y Tecnología de Computadores 

\vspace{0.5cm}

\textbf{Informo:}

\vspace{0.5cm}

Que el presente trabajo, titulado \textit{\textbf{Asistente de voz modular usando APIs libres}},
ha sido realizado bajo mi supervisión por \textbf{Iván Valero Rodríguez}, y autorizo la defensa de dicho trabajo ante el tribunal
que corresponda.

\vspace{0.5cm}

Y para que conste, expiden y firman el presente informe en Granada a Junio de 2022.

\vspace{1cm}

\textbf{El director: }

\vspace{5cm}

\noindent \textbf{D. Pablo García Sanchez}

\chapter*{Agradecimientos}

En primer lugar, me encantaría agradecer a mi tutor (Pablo) por haber podido contar con él para realizar este Trabajo, y haberme dado todo el \textit{feedback} y toda la ayuda posible con cada versión de la memoria, con cada tutoría que se ha realizado y con cada correo que se ha cruzado, pudiendo mejorar este trabajo hasta el final.

También querría darles las gracias a mis padres y mis hermanos por haberme apoyado y por depositar vuestra confianza en mi durante estos años, a pesar de todo.

En cuanto a quienes han estado en estos meses conmigo, me gustaría dar las gracias a mis compañeros de piso (Pepe y David) por haberme soportado estos meses y por los ratos que hemos pasado juntos para despejarnos del día a día, y a Inés y Álex por haberme apoyado y soportado de vez en cuando.

Y para finalizar, querría dar también las gracias a Blanca, por haberme acompañado durante la carrera desde que terminamos el Bachillerato hasta el final, hablando de todo y apoyándome dando igual dónde esta .Y a todos los compañeros y amigos que he conocido durante esta etapa, aportando algo de vida a la estancia en la ETSIIT: la energía de Shei, la simpatía de Noelia y Alba, la calma de Jesús, las risas y consejos con Ignasi, Óscar, Javier, Manu y Miquel... A vosotros y vosotras, muchas gracias por todo.




