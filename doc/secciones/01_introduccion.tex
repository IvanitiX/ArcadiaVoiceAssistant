\chapter{Introducción}

\noindent\fbox{
	\parbox{\textwidth}{
		En esta sección se introducirá los conceptos que se irán desarrollando en el transcurso de este Trabajo Fin de Grado de cara a la definición del problema que se resolverá al realizar el proyecto. Hablaremos de la situación actual y qué puede aportar la temática de este proyecto a ésta.
	}
}

\section{Motivación}

\subsection{La sociedad de la Información y los Asistentes de Voz}
Hoy en día, la Sociedad de la Información está siendo un tema que integramos cada día más en nuestra vida. Desde la introducción de tecnologías como la computación y la telefonía, estos se han ido adaptando a entornos cada vez más generales. 

Actualmente, nuestro día a día está rodeado de sistemas software que gestionan la información relativa a nuestra persona en ámbitos como las administraciones públicas, entornos bancarios e incluso subimos historias en las redes sociales habiendo así, y aún intentando reservarse lo máximo posible, un esquema más o menos complejo de cada persona por las redes.

Además, cada tanto tiempo se publican piezas de \textit{hardware} o programas \textit{software} novedosos que se integran aún más en la vida cotidiana. Como ejemplos podemos tener innovaciones como los teléfonos móviles (quienes evolucionaron en los actuales \textit{smartphones}), los televisores que actualmente disponen de integraciones con los servicios de \textit{streaming}, y lo que en nuestro caso nos compete, que serían los asistentes de voz.

Los asistentes de voz se tratan de sistemas \textit{software} que pueden estar bien como aplicación para sistemas de escritorio, móviles o por Internet; o embebidos en un dispositivo \textit{hardware} que se sitúa en algún punto de la casa; con las cuales se interactúa con la voz, de forma que al hablarle al dispositivo se abra un proceso internamente que toma lo que se habla y devuelve generalmente la reproducción de un audio. Puede ser un sonido, una canción o una voz sintetizada respondiendo a la petición. Trataremos de definir más a detalle las acciones y entresijos que el sistema realiza (desde que recibe el audio hasta que da una respuesta) en el estado del arte.

\subsection{El desarrollo del \textit{Software} y la liberación del código }
\subsubsection{Desarrollando Software}
\subsubsection{Las empresas y la propiedad del código}
\subsubsection{Software de Código Abierto y Software Libre}

\section{Estructura del trabajo}

El trabajo se compone de los siguientes capítulos:

\begin{enumerate}[C{a}pítulo 1.- ]
	\item \textbf{Introducción} Es el capítulo actual, donde 
	\item \textbf{Descripción del problema} Aquí se habla, ya puestos en antecedentes, de qué se quiere resolver con el sistema resuelto
	\item \textbf{Estado del arte} En este capítulo se analizará el estado de la situación del problema, de las soluciones ya realizadas y de las herramientas que se podrían utilizar para ello.
	\item \textbf{Planificación} En este apartado realizaremos una estimación del tiempo necesario para realizar el proyecto.
	\item \textbf{Análisis} En el capítulo se usan las herramientas para preparar el diseño del proyecto de cara a su posterior implementación.
	\item \textbf{Implementación} Tras lo previsto en el anterior epígrafe, aquí se comentará cómo se ha realizado el sistema resultante.
	\item \textbf{Conclusiones} En el último apartado se discutirán tópicos que han podido surgir durante el transcurso del Trabajo, posibles mejoras que podrían implementarse en el futuro y qué se ha aprendido durante el camino que pueda aplicarse en la vida tras el aprendizaje en el Grado.
\end{enumerate}

Este proyecto es software libre, y está liberado con la licencia \cite{gplv3}.