\chapter{Estado del arte}

\section{Asistentes de Voz: Escuchar, procesar, responder}
Los Asistentes de Voz se podrían definir como ++++++++++++++++
\subsection{Una reseña histórica}
\subsubsection{Experimentando con la voz}

\subsubsection{Primeras aplicaciones del reconocimiento de voz}

\subsubsection{Los primeros Asistentes Virtuales: el don artificial del habla}

\subsection{En la actualidad...}
\subsubsection {Aplicaciones populares}

\subsubsection{Un sinfín de formatos}

\section{Software Libre, Software de Código Abierto y Licencias}
El software libre y sus licencias \cite{gplv3} ha permitido llevar a cabo una expansión del
aprendizaje de la informática sin precedentes.
\subsection{¿Desde cuándo existe?}

\subsection{¿En qué beneficia liberar el código?}

\subsection{El curioso mundillo de las Licencias y sus restricciones}

\subsubsection{Detallando el marco legal de las Licencias}

\subsubsection{Free Software Foundation}

\subsubsection{Creative Commons}

\subsubsection{Otros tipos de licencias}

\subsubsection{¿Puedo crear una licencia?}
