\chapter{Descripción del problema}

\noindent\fbox{
	\parbox{\textwidth}{
		En esta sección, una vez introducidos los conceptos que ubican el marco donde se situará el proyecto, se hablará del problema a resolver con el resultado de este Proyecto.
	}
}
\newline

\section{El problema}
Por lo que se ha comentado en la Introducción sobre lo que rodea al desarrollo del Trabajo Fin de Grado, se podríamos encontrar un enunciado para este problema que trataremos de resolver.

\begin{itemize}
	\item Los Asistentes de Voz más populares pertenecen a empresas que dan soluciones con código cerrado, por lo que pueden causar cierto rechazo en algunos sectores de la población. 
	\item Además, desarrollar para tales plataformas puede requerir incluir datos personales o pagos de tasas (mensuales, por tiempo de uso, anuales o de un solo pago), entre otros.
	\item Sin embargo, estos Asistentes son de ayuda y cada día se están integrando a la vida cotidiana de las personas, ya que hay dispositivos que vienen con estos integrados.
	\item Estos pueden servir de ayuda para cierto subgrupo de personas en una sociedad cada vez más digitalizada.
\end{itemize}

\section{La solución propuesta}
La solución que podemos proponer es hacer un software de Asistente de Voz por nuestra cuenta. Esto nos da un roadmap extenso, donde habrá que ver desde un punto de vista competitivo al panorama general para después extraer de ahí la vista general del software, que se reflejará en una serie de diagramas que se reflejarán finalmente en el código.

\section{Las restricciones}

\section{Objetivos}



