\chapter{Planificación}

\noindent\fbox{
	\parbox{\textwidth}{
		En este apartado realizaremos una estimación del tiempo necesario para realizar el proyecto, y cuantificaremos las tareas que se deberían cumplir, acabando con un diagrama de Gantt que contará gráficamente el tiempo empleado en realizarlos aproximadamente.
	}
}

\section{Metodología utilizada}
Para desarrollar el proyecto se ha optado por una Metodología Ágil basada en SCRUM, ya que permite mucha flexibilidad temporal y mayor control en las tareas que hay que realizar. Además, si hay alguna tarea que se debe añadir durante el desarrollo, se puede introducir al backlog y tratarlo posteriormente.

\section{Temporización}
A grandes rasgos, se han desarrollado los objetivos separándolos en acciones más pequeñas para poder cumplirlas. En el análisis se terminarán de definir todas las tareas con tal de poder tener un Product Backlog definido totalmente para poder desarrollar todo lo necesario.

\subsection{El Product Backlog}

\subsection{División en sprints}
Un primer esbozo para dividir todas las tareas en 3 grandes sprints de forma que se vayan realizando de forma continua y constante.

\subsubsection{Sprint 1: ( semanas)}

\subsubsection{Sprint 2: ( semanas)}

\subsubsection{Sprint 3: ( semanas)}

\subsection{Diagrama de Gantt}
La temporización en este punto quedaría desarrollada por el siguiente Diagrama de Gantt:


\section{Seguimiento del desarrollo}
Para seguir el desarrollo se hará uso de dos artefactos que acompañan al framework SCRUM:
\begin{itemize}
	\item Sprint Backlogs : Al principio de cada Sprint se revisará si queda alguna tarea pendiente y si hay que recalcular algo. Como base tendremos los de la subsección 5.2.1
	\item Gráfica de Burndown: Cada vez que se termine una tarea se actualizará esta gráfica.
\end{itemize}

\section{Estimación de costes}
\subsection{El personal}
Si bien este proyecto se va a acabar realizando por una persona realmente, supongamos el caso de que este proyecto lo realizara una pequeña startup de 4 miembros, de forma que hubiera un Scrum Master, un Product Owner, un documentalista y un desarrollador que trabajará principalmente en Python.

\begin{center}
	\begin{table}[H]
		\centering
		\begin{tabularx}{\textwidth}{|X|X|X|X|}
			\hline
			\cellcolor{lightblue}\textbf{Trabajador} & \cellcolor{lightblue}\textbf{Tiempo dedicado (horas)} & \cellcolor{lightblue}\textbf{Coste por hora (€)} & \cellcolor{lightblue}\textbf{Total (€)} \\
			\hline
			Programador Python & 200 & 15 & 3000\\
			\hline
			Documentalista & 50 & 15 & 3000\\
			\hline
			Scrum Master & 20 & 10 & 3000\\
			\hline
			Product Owner & 20 & 10 & 3000\\
			\hline
		\end{tabularx}
		\caption{Tabla de estimación de costes por personal}
	\end{table}
\end{center}
	
\subsection {Costes de productos físicos}
En el inventario hardware del proyecto se usarían dos dispositivos principalmente: Un ordenador portátil donde se desarrolle y pruebe el proyecto, y una Raspberry Pi 3B para las pruebas que se quieran hacer.

Listamos en la siguiente tabla los costes relacionados con ellos:

\begin{center}
	\begin{table}[H]
		\centering
		\begin{tabularx}{\textwidth}{|X|X|X|X|X|}
			\hline
			\cellcolor{lightblue}\textbf{Producto} & 
			\cellcolor{lightblue}\textbf{Vida útil (años)} &  \cellcolor{lightblue}\textbf{Coste (€)} &
			\cellcolor{lightblue}\textbf{Tiempo de implicación en el proyecto (meses)} &
			\cellcolor{lightblue}\textbf{Coste por uso (€)} \\
			\hline
			Raspberry Pi 3B & 8 & 35 & 1 & 1\\
			\hline
			Portátil HP Pavilion Gaming 15-2008ec & 4 & 550 & 4 & 1\\
			\hline
		\end{tabularx}
		\caption{Tabla de estimación de costes por productos físicos}
	\end{table}
\end{center}

\subsection{Costes de productos lógicos}
En el caso de las herramientas software, notamos que todas son gratuitas, pero debido a que son unas cuantas y con propósitos variados, vemos interesante desglosarlos.

\begin{center}
	\begin{table}[H]
		\centering
		\begin{tabularx}{\textwidth}{|l|X|X|}
			\hline
			\cellcolor{lightblue}\textbf{Producto} & \cellcolor{lightblue}\textbf{Descripción} & \cellcolor{lightblue}\textbf{Coste (en €)} \\
			\hline
			Visual Studio Code & Editor de Código. Soporta integraciones para Python, entre otros lenguajes & 0 \\
			\hline
			Vosk & Framework de Código Abierto para Reconocimiento de Voz & 0 \\
			\hline
			eSpeak & Framework de Código Abierto para Text-to-Speech & 0 \\
			\hline
			TeX Studio & Editor de \LaTeX  para redactar la documentación & 0 \\
			\hline
			Draw.io & Editor de Diagramas en la nube & 0 \\
			\hline
			GitHub & Almacén para el repositorio, basado en Git. La visibilidad privada está integrada en el \textit{Students Pack} & 0 \\
			\hline
			GanttProject & Editor de Diagramas de Gantt & 0 \\
			\hline
			& \textbf{Total} & \textbf{0} \\
			\hline
		\end{tabularx}
		\caption{Tabla de estimación de costes por productos lógicos}
	\end{table}
\end{center}

En total, nos daríamos con el siguiente presupuesto:

\begin{center}
	\begin{table}[H]
		\centering
		\begin{tabularx}{10cm}{|X|X|}
			\hline
			\cellcolor{lightblue}\textbf{Concepto} & \cellcolor{lightblue}\textbf{Total (€)}  \\
			\hline
			Personal & 200 \\
			\hline
			Productos físicos & 50 \\
			\hline
			Productos lógicos & 0\\
			\hline
			Total & 20\\
			\hline
		\end{tabularx}
		\caption{Tabla de estimación de costes totales}
	\end{table}
\end{center}


