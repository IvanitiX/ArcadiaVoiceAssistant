%%%%%%%%%%%%%%%%%%%%%%%%%%%%%%%%%%%%%%%%%
% Short Sectioned Assignment LaTeX Template Version 1.0 (5/5/12)
% This template has been downloaded from: http://www.LaTeXTemplates.com
% Original author:  Frits Wenneker (http://www.howtotex.com)
% License: CC BY-NC-SA 3.0 (http://creativecommons.org/licenses/by-nc-sa/3.0/)
%%%%%%%%%%%%%%%%%%%%%%%%%%%%%%%%%%%%%%%%%

% \documentclass[paper=a4, fontsize=11pt]{scrartcl} % A4 paper and 11pt font size
\documentclass[12pt, a4paper, openany]{book}
\usepackage[T1]{fontenc} % Use 8-bit encoding that has 256 glyphs
\usepackage{fourier} % Use the Adobe Utopia font for the document - comment this line to return to the LaTeX default
\usepackage[utf8]{inputenc}
\usepackage{listings} % para insertar código con formato similar al editor
\usepackage[spanish, es-tabla]{babel} % Selecciona el español para palabras introducidas automáticamente, p.ej. "septiembre" en la fecha y especifica que se use la palabra Tabla en vez de Cuadro
\usepackage{url} % ,href} %para incluir URLs e hipervínculos dentro del texto (aunque hay que instalar href)
\usepackage{graphics,graphicx, float} %para incluir imágenes y colocarlas
\usepackage[gen]{eurosym} %para incluir el símbolo del euro
\usepackage{cite} %para incluir citas del archivo <nombre>.bib
\usepackage{enumerate}
\usepackage{hyperref}
\usepackage{graphicx,subfig}
\usepackage{tabularx, multirow}
\usepackage{booktabs}
\usepackage[table,xcdraw]{xcolor}

\hypersetup{
	colorlinks=true,	% false: boxed links; true: colored links
	linkcolor=black,	% color of internal links
	urlcolor=cyan		% color of external links
}
%\renewcommand{\familydefault}{\sfdefault}
\usepackage{fancyhdr} % Custom headers and footers
\pagestyle{fancyplain} % Makes all pages in the document conform to the custom headers and footers
\fancyhead[EL]{Asistente de Voz Modular usando APIs libres} % Empty left header
\fancyhead[ER]{} % Empty left header
\fancyhead[C]{} % Empty center header
\fancyhead[OL]{} % My name
\fancyhead[OR]{Iván Valero Rodríguez} % My name
\fancyfoot[L]{} % Empty left footer
\fancyfoot[C]{} % Empty center footer
\fancyfoot[R]{\thepage} % Page numbering for right footer
%\renewcommand{\headrulewidth}{0pt} % Remove header underlines
\renewcommand{\footrulewidth}{0pt} % Remove footer underlines
\setlength{\headheight}{13.6pt} % Customize the height of the header

\usepackage{titlesec, blindtext, color}
\definecolor{gray75}{gray}{0.75}
\definecolor{azul}{HTML}{456CF7}
\definecolor{lightblue}{HTML}{d5d9e6}
\definecolor{mintgreen}{HTML}{c2edc5}
\newcommand{\hsp}{\hspace{20pt}}
\setcounter{secnumdepth}{4}
\usepackage[Bjornstrup]{fncychap}
\ChNumVar{\fontsize{76}{80}\usefont{OT1}{pzc}{m}{n}\selectfont}
\ChTitleVar{\raggedleft\Large\bfseries}

\renewcommand\DOCH{%
  \settowidth{\py}{\CNoV\thechapter}
  \addtolength{\py}{-10pt}
  \fboxsep=0pt%
  \colorbox{lightblue}{\rule{0pt}{40pt}\parbox[b]{\textwidth}{\hfill}}%
  \kern-\py\raise20pt%
  \hbox{\color{azul}\CNoV\thechapter}\\%
}

\renewcommand\DOTI[1]{%
  \nointerlineskip\raggedright%
  \fboxsep=\myhi%
  \vskip-1ex%
  \colorbox{lightblue}{\parbox[t]{\mylen}{\CTV\FmTi{#1}}}\par\nobreak%
  \vskip 40pt%
}

\renewcommand\DOTIS[1]{%
  \fboxsep=0pt
  \colorbox{mintgreen}{\rule{0pt}{40pt}\parbox[b]{\textwidth}{\hfill}}\\%
  \nointerlineskip\raggedright%
  \fboxsep=\myhi%
  \colorbox{mintgreen}{\parbox[t]{\mylen}{\CTV\FmTi{#1}}}\par\nobreak%
  \vskip 40pt%
 }



\begin{document}

	% Plantilla portada UGR
	\begin{titlepage}
\newlength{\centeroffset}
\setlength{\centeroffset}{-0.5\oddsidemargin}
\addtolength{\centeroffset}{0.5\evensidemargin}
\thispagestyle{empty}

\noindent\hspace*{\centeroffset}\begin{minipage}{\textwidth}

\centering
\includegraphics[width=0.9\textwidth]{logos/logo_ugr.jpg}\\[1.4cm]

\textsc{ \Large TRABAJO FIN DE GRADO\\[0.2cm]}
\textsc{ GRADO EN INGENIERÍA INFORMÁTICA}\\[1cm]

{\Huge\bfseries Asistente de voz modular \\}
\noindent\rule[-1ex]{\textwidth}{3pt}\\[3.5ex]
{\large\bfseries usando APIs libres }
\end{minipage}

\vspace{2.5cm}
\noindent\hspace*{\centeroffset}
\begin{minipage}{\textwidth}
\centering

\textbf{Autor}\\ {Iván Valero Rodríguez}\\[2.5ex]
\textbf{Director}\\ {Pablo García Sánchez}\\[2cm]
\includegraphics[width=0.3\textwidth]{logos/etsiit_logo.png}\\[0.1cm]
\textsc{Escuela Técnica Superior de Ingenierías Informática y de Telecomunicación}\\
\textsc{---}\\
Granada, Julio de 2022
\end{minipage}
\end{titlepage}


	% Plantilla prefacio UGR
	\thispagestyle{empty}

\begin{center}
{\large\bfseries Asistente de voz modular \\ usando APIs libres }\\
\end{center}
\begin{center}
Iván Valero Rodríguez\\
\end{center}

%\vspace{0.7cm}

\vspace{0.5cm}
\noindent{\textbf{Palabras clave}: \textit{software libre}
\vspace{0.7cm}

\noindent{\textbf{Resumen}\\
	

\cleardoublepage

\begin{center}
	{\large\bfseries Voice assistant using Open Source \\ and Free Software APIs}\\
\end{center}
\begin{center}
	Iván Valero Rodríguez\\
\end{center}
\vspace{0.5cm}
\noindent{\textbf{Keywords}: \textit{open source}, \textit{floss}
\vspace{0.7cm}

\noindent{\textbf{Abstract}\\


\cleardoublepage

\thispagestyle{empty}

\noindent\rule[-1ex]{\textwidth}{2pt}\\[4.5ex]

D. \textbf{Tutora/e(s)}, Profesor(a) del ...

\vspace{0.5cm}

\textbf{Informo:}

\vspace{0.5cm}

Que el presente trabajo, titulado \textit{\textbf{Chief}},
ha sido realizado bajo mi supervisión por \textbf{Estudiante}, y autorizo la defensa de dicho trabajo ante el tribunal
que corresponda.

\vspace{0.5cm}

Y para que conste, expiden y firman el presente informe en Granada a Junio de 2022.

\vspace{1cm}

\textbf{El/la director(a)/es: }

\vspace{5cm}

\noindent \textbf{(nombre completo tutor/a/es)}

\chapter*{Agradecimientos}






	% Índice de contenidos
	\newpage
	\tableofcontents

	% Índice de imágenes y tablas
	\newpage
	\listoffigures

	% Si hay suficientes se incluirá dicho índice
	\listoftables 
	\newpage

	% Introducción
	\newpage 
	\chapter{Introducción}

\noindent\fbox{
	\parbox{\textwidth}{
		En esta sección se introducirá los conceptos que se irán desarrollando en el transcurso de este Trabajo Fin de Grado de cara a la definición del problema que se resolverá al realizar el proyecto. Hablaremos de la situación actual y qué puede aportar la temática de este proyecto a ésta.
	}
}

\section{Motivación}

\subsection{La sociedad de la Información y los Asistentes de Voz}
Hoy en día, la Sociedad de la Información está siendo un tema que integramos cada día más en nuestra vida. Desde la introducción de tecnologías como la computación y la telefonía, estos se han ido adaptando a entornos cada vez más generales. 

Actualmente, nuestro día a día está rodeado de sistemas software que gestionan la información relativa a nuestra persona en ámbitos como las administraciones públicas, entornos bancarios e incluso subimos historias en las redes sociales habiendo así, y aún intentando reservarse lo máximo posible, un esquema más o menos complejo de cada persona por las redes.

Además, cada tanto tiempo se publican piezas de \textit{hardware} o programas \textit{software} novedosos que se integran aún más en la vida cotidiana. Como ejemplos podemos tener innovaciones como los teléfonos móviles (quienes evolucionaron en los actuales \textit{smartphones}), los televisores que actualmente disponen de integraciones con los servicios de \textit{streaming}, y lo que en nuestro caso nos compete, que serían los asistentes de voz.

Los asistentes de voz se tratan de sistemas \textit{software} que pueden estar bien como aplicación para sistemas de escritorio, móviles o por Internet; o embebidos en un dispositivo \textit{hardware} que se sitúa en algún punto de la casa; con las cuales se interactúa con la voz, de forma que al hablarle al dispositivo se abra un proceso internamente que toma lo que se habla y devuelve generalmente la reproducción de un audio. Puede ser un sonido, una canción o una voz sintetizada respondiendo a la petición. 

% ¿Y qué aportan los Asistentes de Voz a la Sociedad de la Información?

Trataremos de definir más a detalle las acciones y entresijos que el sistema realiza (desde que recibe el audio hasta que da una respuesta) en el estado del arte.

\subsection{El desarrollo del \textit{Software} y la propiedad del código }
\subsubsection{Desarrollando Software}
Haciendo mención al profesor Alberto Espinosa \cite{Prieto2006-ss}, podríamos entender el software no sólo como aquellos programas que se pueden ejecutar en un computador, sino además a todo lo que lo relaciona, desde las metodologías empleadas en su desarrollo, los datos que internamente interpretan y cómo los organizan, el análisis de cómo se implementará el sistema resultante y, en general, toda herramienta que permita su realización. 

El proceso de hacer Software ha ido cambiando con el pasar de los años con tal de hacer más eficiente el proceso de diseño y codificación de este. Se pasó de un procedimiento lineal en cascada con una codificación sin estandarizar, a una serie de aproximaciones que todos los desarrolladores adoptan; basadas en estilos de codificación, metodologías ágiles como Scrum y XP, o patrones de diseño y arquitecturales; entre otros. 

Por consiguiente, se lleva a una comprensión del código y de cómo integrar el sistema o interactuar con éste, una documentación sobre todo lo que se haya implicado con tal de converger en el producto resultante.

Por lo que se ha expuesto entonces en este epígrafe, vemos que al trabajar en el área del Desarrollo del Software trabajamos con convenciones estandarizadas que acaban en una serie de documentos que exponen cómo se ha ideado el programa o programas a programar junto con instrucciones para poder compilarlo y/o instalarlo; el código ya programado (en general con comentarios) y el ejecutable o sistema montado (si se aplica)

\subsubsection{Las empresas y la propiedad del código}
Las empresas llevan adaptándose o adelantándose a los hábitos que la sociedad va experimentando día a día.
Desde la introducción de los computadores (primero en el ámbito científico y con el pasar de los años extendiéndose a un público más general), las empresas requieren de programas que por les faciliten la gestión y/o automatización de algunos ámbitos y procesos que aplican entre sus operaciones y responsabilidades.

Por consecuencia, se da una necesidad de preparar más programas para las terminales que satisfagan las tareas más mecánicas y repetitivas o que sirvan de ayuda a la realización de tal tarea consiguiendo más eficiencia en el proceso. Para suplir esa demanda se crean otros tipos de empresa dedicados a desarrollar ese software (a medida o para un grupo de usuarios más extendido) consiguiendo de paso algún beneficio económico. Entre ellos podríamos destacar a Microsoft, Apple o IBM, quienes en los últimos años han ido tomando relevancia en conjunto con la historia reciente de la computación y de la llegada del ordenador personal.

Eso sí, si estas empresas toman tiempo y recursos en realizar soluciones computacionales para satisfacer esas demandas con tal de conseguir beneficios, ¿cómo pueden protegerse de que la competencia pueda sacar fácilmente una versión con las mismas o similares prestaciones, sacando así una versión de esa solución por un precio menor? Una de las ideas más intuitivas sería simplemente \textbf{no dando el código} ni las instrucciones para montarlo, dando en su lugar sólo un ejecutable que pueda comprender el Sistema Operativo que se use. 

Si bien es una solución correcta, posee ciertos inconvenientes que podrían acarrear el desarrollo a largo plazo. Podríamos listar unas cuantas:

\begin{itemize}
	\item \textbf{Compatibilidad}: Los programas resultantes están compilados y preparados para ser ejecutados en los entornos que la empresa que los desarrolla ve a conveniencia del usuario final. Sin embargo, lo que a tales creadores les parece el entorno medio para poder usar su software; puede acabar no coincidiendo con el real. Y aún así, si logra que funcione para los usuarios de ese entorno, puede que otros que no trabajen en tal contexto no puedan hacer usufructo de la aplicación aún habiendo pagado por ella.
	
	\item \textbf{Privacidad y auditoría}: Cada día, la sociedad empieza a usar dispositivos para poder acceder a la información que lo rodea. Así mismo, desean poder tener protegidos los datos que le definen como persona en el mundo real y poder aislarlos de su perfil en las redes. Si bien hay normativas y organismos que velan para que sus datos sean lo más confidenciales posible (por ejemplo, en \cite{lopd-2018}) ; al tener un código cerrado se desconoce realmente si las empresas obran de buena fe y si todos los datos y telemetría que recogen (en caso de que proceda) son los descritos en sus Términos y Condiciones de Uso.
	
	Ante ello, se podrían exigir auditorías de código por parte de entes reguladores o de terceros para poder analizar lo que hace el programa, qué datos usa y cómo los trata, para poder compararlos con los descritos en el documento redactado por la empresa y poder decidir la elección de usar o no el programa.
	
	\item \textbf{Restricción de la libertad de uso}
\end{itemize}

%Hablar de los términos legales y la Propiedad Intelectual

\subsubsection{Software de Código Abierto y Software Libre}

En respuesta a los inconvenientes discutidos en la anterior enumeración, han surgido iniciativas como la \textit{Free Software Foundation} \cite{fsf-about}, donde se promueve que el código del programa resultante esté disponible al público junto con la documentación necesaria con tal de incentivar la colaboración entre programadores de tener aplicaciones para todos, que permitan ser modificados sin restricciones.

%Hablar de las 4 libertades

Si bien la corriente ideológica de la FSF establece muy bien los fundamentos para luchar contra el cierre del código, esa ideología impone una serie de libertades y reglas estrictas para que se tome como tal. 

En respuesta, y como una iniciativa más abierta que la comentada previamente, surge la \textit{Open Source Initiative} \cite{osi-about}, quienes definen una iniciativa más centrada en pragmas, permitiendo así una liberación del código a la vez que se permite una protección de la autoría ligeramente mayor. De esta manera, empresas como Google y Facebook permiten que sus proyectos puedan recibir colaboración de la comunidad informática a la vez que pueden restringir el ámbito de uso que concede ese código o su software resultante.

%Hablar de los términos legales en las licencias.

Detallaremos en el Estado del Arte un poco más sobre las licencias existentes que puedan ser más relevantes, las diferencias entre ellas y posibles cuestiones con respecto a usar varios proyectos pequeños para realizar una aplicación más grande.

\section{Estructura del trabajo}

El trabajo se compone de los siguientes capítulos:

\begin{enumerate}[C{a}pítulo 1.- ]
	\item \textbf{Introducción} Es el capítulo actual, donde se explica los pilares que cimientan el proyecto para motivar la realización de este Trabajo.
	\item \textbf{Descripción del problema} Aquí se habla, ya puestos en antecedentes, de qué se quiere resolver con el sistema resuelto
	\item \textbf{Estado del arte} En este capítulo se analizará la situación del problema, las soluciones ya realizadas y las herramientas que se podrían utilizar para crear la nuestra propia.
	\item \textbf{Planificación} En este apartado realizaremos una estimación del tiempo necesario para realizar el proyecto.
	\item \textbf{Análisis} En el capítulo se usan las herramientas para preparar el diseño del proyecto de cara a su posterior codificación.
	\item \textbf{Implementación} Tras lo previsto en el anterior epígrafe, aquí se comentará cómo se ha realizado el sistema resultante.
	\item \textbf{Conclusiones} En el último apartado se discutirán tópicos que han podido surgir durante el transcurso del Trabajo, posibles mejoras que podrían implementarse en el futuro, qué se ha aplicado en las diferentes asignaturas del Grado y qué se ha aprendido durante el camino que pueda aplicarse en la vida tras esta etapa de aprendizaje.
\end{enumerate}

Este proyecto es software libre, y está liberado con la licencia \cite{gplv3}.

	% Descripción del problema y hasta donde se llega
	\chapter{Descripción del problema}

\noindent\fbox{
	\parbox{\textwidth}{
		En esta sección, una vez introducidos los conceptos que ubican el marco donde se situará el proyecto, se hablará del problema a resolver con el resultado de este Proyecto.
	}
}
\newline

\section{El problema}
Por lo que se ha comentado en la Introducción sobre lo que rodea al desarrollo del Trabajo Fin de Grado, se podríamos encontrar un enunciado para este problema que trataremos de resolver.

\begin{itemize}
	\item Los Asistentes de Voz más populares pertenecen a empresas que dan soluciones con código cerrado, por lo que pueden causar cierto rechazo en algunos sectores de la población. 
	\item Además, desarrollar para tales plataformas puede requerir incluir datos personales o pagos de tasas (mensuales, por tiempo de uso, anuales o de un solo pago), entre otros.
	\item Sin embargo, estos Asistentes son de ayuda y cada día se están integrando a la vida cotidiana de las personas, ya que hay dispositivos que vienen con estos integrados.
	\item Estos pueden servir de ayuda para cierto subgrupo de personas en una sociedad cada vez más digitalizada.
\end{itemize}

\section{La solución propuesta}
La solución que podemos proponer es hacer un software de Asistente de Voz por nuestra cuenta. Esto nos da un roadmap extenso, donde habrá que ver desde un punto de vista competitivo al panorama general para después extraer de ahí la vista general del software, que se reflejará en una serie de diagramas que se reflejarán finalmente en el código.

Se explorará la posibilidad de modularizar el Asistente con tal de poder agregar funcionalidades (en forma de preguntas, respuestas y utilidades) de forma sencilla.

Otro tema interesante de tratar sería a qué entorno dirigir este Proyecto. En principio se hará para Sistemas Operativos de escritorio, aunque si el tiempo lo permite se explorará integrarlo en un SBC como podría ser una Raspberry Pi sin pantalla, permitiendo su funcionamiento a través de un micrófono y un altavoz. 

\section{Las restricciones}
\begin{itemize}
	\item El más importante de todos, se tratarán de usar APIs libres u Open Source, lo máximo que se pueda. Será la gran diferencia con respecto a las otras alternativas más populares.
	\item Por consecuencia sobre el anterior punto, la licencia a aplicar al conjunto del proyecto deberá ser compatible con la serie de complementos y componentes que se usen. Ello implica que el producto resultante será de Código Abierto o Libre.
	\item El proyecto deberá realizarse antes de la fecha de entrega estimada, siendo esta Junio de 2022.
\end{itemize}

\section{Objetivos}
Para llegar a la solución propuesta, se emplearán herramientas y se requerirá investigación y desarrollo que podríamos definir en base a una serie de objetivos de un granulado más fino. Además se separarán estos objetivos en dos tipos: 
\begin{itemize}
	\item \textbf{De Investigación y Aprendizaje (O-IA)} Son aquellos objetivos donde se involucra una formación sobre tecnologías, métodos y herramientas para tener una visión global de la situación y de cómo facilitar el trabajo durante la fase de Diseño y Desarrollo.
	\item \textbf{De Diseño y Desarrollo (O-DD)} Son aquellas metas que se resolverán mediante las herramientas de planteamiento del sistema y su posterior codificación, pruebas y posible producción.
\end{itemize}

\subsection{Objetivos de Investigación y Aprendizaje}

\begin{enumerate}[O-{IA}.1 -]
	\item Entender el funcionamiento de los Asistentes de Voz.
	\item Analizar el panorama del campo de los Asistentes de Voz y los avances que se han realizado.
	\item Analizar las prestaciones que ofrecen otras alternativas propietarias y libres (si las hay)
	\item Sintetizar los posibles componentes que conforman un Asistente de Voz y la escalabilidad de cada uno de ellos.
	\item Entender el impacto del software propietario y sus alternativas libres o de Código Abierto.
	\item Analizar las distintas licencias que se ofrecen y las compatibilidades entre éstas para elegir la idónea para la aplicación resultante.
	\item Analizar varias aproximaciones de desarrollo (desarrollo en cascada, Metodologías Ágiles...) para elegir la planificación temporal y el enfoque del Trabajo.
	\item Analizar el impacto que los descubrimientos en los Asistentes de Voz pueden producir en la Sociedad de la Información.
	\item Leer sobre aquellos estándares y convenciones para facilitar la compartición y liberación del proyecto (por ejemplo. Códigos de conducta, Guías de contribución)
	\item Aprender sobre herramientas de \textit{Continous Integration/Continous Deployment} y entender cómo pueden estas acelerar el desarrollo.
\end{enumerate}

\subsection{Objetivos de Diseño y Desarrollo}

\begin{enumerate}[O-DD.1 -]
	\item Pensar en perfiles de personas que usarían el sistema
	\item Diseñar casos de uso en los que el software podría presentar algún problema con tal de buscar soluciones o limitaciones.
	\item Diseñar las clases que albergarán los componentes de un Asistente de Voz, su interacción con los periféricos de Entrada/Salida y con aquellas APIs externas que se requieran para el funcionamiento básico del programa.
	\item Idear una vía para escalar el número de posibles frases e interacciones que pueda reconocer el proyecto.
	\item Planear la arquitectura del proyecto resultante, siguiendo las buenas prácticas del Desarrollo de Software
	\item Con base a la arquitectura, codificar lo necesario para la implementación de la aplicación.
	\item Diseñar pruebas para asegurar la calidad del proyecto y conocer los límites de la codificación.
	\item Redactar documentación referida a la arquitectura del proyecto para su posterior mantenimiento
	\item Redactar documentación orientada a aquellos desarrolladores que deseen escalar las oraciones que pueda reconocer y las respuestas que pueda contestar el Asistente de Voz resultante.
	
\end{enumerate}




	% Estado del arte
	% 	1. Crítica al estado del arte
	% 	2. Propuesta
	\chapter{Estado del arte}

\section{Asistentes de Voz: Escuchar, procesar, responder}
Los Asistentes de Voz se podrían definir como ++++++++++++++++
\subsection{Una reseña histórica}
\subsubsection{Experimentando con la voz}

\subsubsection{Primeras aplicaciones del reconocimiento de voz}

\subsubsection{Los primeros Asistentes Virtuales: el don artificial del habla}

\subsection{En la actualidad...}
\subsubsection {Aplicaciones populares}

\subsubsection{Un sinfín de formatos}

\section{Software Libre, Software de Código Abierto y Licencias}
El software libre y sus licencias \cite{gplv3} ha permitido llevar a cabo una expansión del
aprendizaje de la informática sin precedentes.
\subsection{¿Desde cuándo existe?}

\subsection{¿En qué beneficia liberar el código?}

\subsection{El curioso mundillo de las Licencias y sus restricciones}

\subsubsection{Detallando el marco legal de las Licencias}

\subsubsection{Free Software Foundation}

\subsubsection{Creative Commons}

\subsubsection{Otros tipos de licencias}

\subsubsection{¿Puedo crear una licencia?}

	
	\newpage 
	\chapter{Especificación de requisitos}

\noindent\fbox{
	\parbox{\textwidth}{
		En esta sección se tratarán de recoger todos los detalles que definan y limiten el funcionamiento de la aplicación.
	}
}

\section{Personas: ¿Quienes lo podrían usar?}

\section{Casos de uso: ¿Qué podrían hacer?}

\section{La Ingeniería de Requisitos en acción}

\subsection{Requisitos funcionales}

\subsection{Requisitos no funcionales}
	
	\newpage 
	\chapter{Planificación}

\noindent\fbox{
	\parbox{\textwidth}{
		En este apartado realizaremos una estimación del tiempo necesario para realizar el proyecto, y cuantificaremos las tareas que se deberían cumplir, acabando con un diagrama de Gantt que contará gráficamente el tiempo empleado en realizarlos aproximadamente.
	}
}

\section{Metodología utilizada}
Para desarrollar el proyecto se ha optado por una Metodología Ágil basada en SCRUM, ya que permite mucha flexibilidad temporal y mayor control en las tareas que hay que realizar. Además, si hay alguna tarea que se debe añadir durante el desarrollo, se puede introducir al backlog y tratarlo posteriormente.

\section{Temporización}
A grandes rasgos, se han desarrollado los objetivos separándolos en acciones más pequeñas para poder cumplirlas. En el análisis se terminarán de definir todas las tareas con tal de poder tener un Product Backlog definido totalmente para poder desarrollar todo lo necesario.

\subsection{El Product Backlog}

\subsection{División en sprints}
Un primer esbozo para dividir todas las tareas en 3 grandes sprints de forma que se vayan realizando de forma continua y constante.

\subsubsection{Sprint 1: ( semanas)}

\subsubsection{Sprint 2: ( semanas)}

\subsubsection{Sprint 3: ( semanas)}

\subsection{Diagrama de Gantt}
La temporización en este punto quedaría desarrollada por el siguiente Diagrama de Gantt:


\section{Seguimiento del desarrollo}
Para seguir el desarrollo se hará uso de dos artefactos que acompañan al framework SCRUM:
\begin{itemize}
	\item Sprint Backlogs : Al principio de cada Sprint se revisará si queda alguna tarea pendiente y si hay que recalcular algo. Como base tendremos los de la subsección 5.2.1
	\item Gráfica de Burndown: Cada vez que se termine una tarea se actualizará esta gráfica.
\end{itemize}

\section{Estimación de costes}
\subsection{El personal}
Si bien este proyecto se va a acabar realizando por una persona realmente, supongamos el caso de que este proyecto lo realizara una pequeña startup de 4 miembros, de forma que hubiera un Scrum Master, un Product Owner, un documentalista y un desarrollador que trabajará principalmente en Python.

\subsection {Costes de productos físicos}
En el inventario hardware del proyecto se usarían dos dispositivos principalmente: Un ordenador portátil donde se desarrolle y pruebe el proyecto, y una Raspberry Pi 3B para las pruebas que se quieran hacer.

Listamos en la siguiente tabla los costes relacionados con ellos:

\subsection{Costes de productos lógicos}
En el caso de las herramientas software, notamos que todas son gratuitas, pero debido a que son unas cuantas y con propósitos variados, vemos interesante desglosarlos.




	% Análisis del problema
	% 1. Análisis de requisitos
	% 2. Análisis de las soluciones
	% 3. Solucion propuesta
	% 4. Análisis de seguridad
	\newpage 
	\chapter{Análisis del problema}

\noindent\fbox{
	\parbox{\textwidth}{
		En el capítulo se usan las herramientas para preparar el diseño del proyecto de cara a su posterior codificación.
	}
}

\section{Comparando APIs de Reconocimiento de Voz}
Como hablamos en el Capítulo 2, hoy en día existen herramientas que reconocen la voz. Algunas son de software propietario o de Software como Servicio (por ejemplo, IBM Watson SR o Google SR), pero también hay proyectos de software de código abierto que nos permiten trabajar con ello sin tener que saber los tópicos de la Inteligencia Artificial a fondo, trabajando de forma transparente al desarrollador. 

Entre estos proyectos podemos encontrar dos:
\begin{itemize}
	\item \textbf{Vosk API}:Desarrollado por Alpha Cephei en 2019 como wrapper de Kaldi , es un sistema de Reconocimiento de Voz offline que actualmente soporta 17 lenguajes y destaca por funcionar en dispositivos más limitados como la Raspberry Pi. Sus modelos más básicos son de 50 MB, pero se pueden adaptar a modelos más complejos, consiguiendo así una escalabilidad en el sistema. Además, soporta reconocimiento del habla. En su arquitectura se usa una red neuronal en conjunto con un Modelo Oculto de Markov. Usa una licencia Apache 2.0
	
	\item \textbf{Mozilla DeepSpeech}: Es una API de Reconocimiento de Voz realizado por Mozilla desde 2016, basado en un paper de Baidu sobre un sistema de Reconocimiento del Habla usando algoritmos de Deep Learning en conjunto con optimizaciones para lograr resultados más rápidos usando múltiples GPU para alimentar una Red Neuronal Recurrente que modelizara un lenguaje en base a ingentes cantidades de datos. Por tanto, acaban dando en la API una interfaz para poder usarlo en nuestros ordenadores y una serie de modelos optimizados gracias al entrenamiento que pueden realizar.
	
\end{itemize}

Para saber cuál sería la API que más nos interesara, podríamos montar un experimento donde evaluaríamos:
\begin{itemize}
	\item La facilidad de implementación
	\item La precisión de los modelos y de la API.
\end{itemize}

Este experimento se basaría en coger a varios sujetos que leyeran un mismo texto.
Esas lecturas se pasarían por un programa que emplea la API de forma que acabamos convirtiendo la voz en texto. Tras ello, se comparan ambos textos para ver el \textbf{Word Error Rate (WER)}, una métrica que mide la precisión entre lo que se ha querido decir y lo que un Reconocedor del Habla transcribe. 

El algoritmo del Word Error Rate consiste en mapear dos cadenas de texto (una indica qué se quiere decir y la otra es lo que se ha detectado) para extraer las diferencias (palabras \textbf{\textit{I}}ntroducidas, \textbf{\textit{S}}ustituidas o \textbf{\textit{R}}etiradas)  y medirlas respecto al \textbf{\textit{N}}úmero total de palabras del texto. Por tanto, su fórmula sería:

\begin{equation}
	WER = \frac{I+S+R}{N}
\end{equation}

También estaría interesante fijarnos en el CER (Character Error Rate) para ver también a nivel de caracteres qué errores se pueden encontrar en la transcripción (por ejemplo, los acentos). La fórmula sería la misma pero las inserciones, sustituciones y eliminaciones se observan letra a letra.

\begin{equation}
	CER = \frac{I_c+S_c+R_c}{N_c}
\end{equation}

Nos quedaríamos entonces con aquella implementación que sea más fácil de usar en conjunto con el modelo más preciso. 

Otra cosa que también podemos probar, de paso, es decir varios nombres propios con tal de mirar si le podemos poner nombre al Asistente, pues hoy en día es común usar como \textit{trigger word} un nombre propio (como \textit{Alexa}, \textit{Siri} o \textit{Cortana}). También podemos probar a decir acciones que debería reconocer el asistente habitualmente para funcionalidades futuras, incluso probar a decir cosas en otros idiomas (por ejemplo, canciones con títulos en inglés) para ver cómo reacciona.

Así pues, se ha redactado el siguiente texto para hacer las pruebas:

\noindent\fbox{
	\parbox{\textwidth}{
		Lúmina y Arcadia se encuentran en un portal a medio camino entre sus casas, cerca del centro comercial. Ellas habían quedado para ver la nueva película que tanto promocionaban por la tele y buscaron un momento entre sus agendas para ir a verla. 
		
		Lumi comparte uno de sus auriculares con su amiga y  enciende el móvil. Entra a Spotify y reproduce un tema de Katy Perry mientras pasean hacia su destino.
		
		Al llegar, buscan el cine y compran las entradas y las palomitas. Al entrar a la sala ven un temporizador de 5 minutos en la pantalla, que además no paraba de poner publicidad.
		
		De repente, todo se para por un corte de luz. Todo estaba a oscuras y alguien enciende la luz de su teléfono, aunque no sirve de mucho hasta que vuelve el suministro a funcionar y ya puedan ver la película.
		
		Al terminar, miran qué tiempo hace para saber si esperar a que les recojan o volver andando. Aunque lo piensan mejor tras mirar qué hora es y deciden ir a tomar un café antes de irse.
	}
}

Además de este texto tan largo, se ha probado a hacer audios más cortos, ya que suelen estar entrenados con segmentos de pocos segundos, a modo de órdenes:

\noindent\fbox{
	\parbox{\textwidth}{
		Enciende la luz.
		
		Ponme esa canción que tanto me gusta.
		
		Repite lo que diga.
		
		Para.
		
		Espera un momento.
		
		Reproduce el informativo de la mañana.
		
		Ponme una alarma a las ocho y media de la mañana.
		
		Prepara un temporizador de 5 minutos.
	}
}

De cada frase se preparará un par de variantes, ya que queremos ver cómo se comportan al nombrar varias propuestas de nombre para el producto. Estos nombres serían Arcadia y Lúmina.

En primera instancia, los cuales han grabado su voz durante la lectura. Se ha comprobado previamente de forma manual que lo que se oye es exactamente lo que se lee, ya que podría ocasionar cierto ruido el hecho de que no fuera así. Tras ello se ha tratado el audio con cada uno de los modelos y configuraciones.

En el caso de Vosk, acabaremos extrayendo su WER y CER para cada uno de los audios según su modelo oficial.

Para el caso de DeepSpeech, se ha trabajado con dos modelos ya entrenados y dos scorers, combinando ambos parámetros.

Los modelos que se probarán son:
\begin{itemize}
	\item \textbf{Polyglot}: Este modelo fue entrenado por un usuario de la Comunidad de Mozilla, entrenando un set de 812 horas con GPUs de Nvidia \cite{scribosermo}. Usa una licencia LGPL versión 3.
	\item \textbf{Rhasspy}: Este modelo estaba pensado para usarse en otro sistema de asistente de voz con una Raspberry Pi, basado en una versión anterior del modelo descrito anteriormente.
\end{itemize}

En el mismo modelo Polyglot, se nos daban dos scorers. Un scorer es un complemento del modelo del lenguaje que permite predecir qué palabra sigue. En este caso, se ha decidido usar dos para ver si causaban alguna diferencia:

\begin{itemize}
	\item \textbf{Ken}: Es un scorer ligero, con poco entrenamiento.
	\item \textbf{Poco} : Es un scorer pesado pero más completo.
\end{itemize} 

DeepSpeech también admitía dos valores extra: Alfa y Beta. Estos parámetros definen el peso del modelo del lenguaje y cuánto se penaliza por introducir una palabra. Para tener un abanico de opciones, para cada combinación se ha tratado de buscar un valor óptimo para estos parámetros, con la precisión de una décima, en un rango de 1 a 2 para el valor Alfa y un rango de 0 a 1 para el valor Beta.

Agrupamos todos los valores en esta tabla:

{\small\tabcolsep=2pt\setlength\LTleft{-35pt}	
	\begin{xltabular}{\textwidth}{|ll|l|llll|}
		\hline \multicolumn{4}{|r|}{{Continúa en la siguiente página $>>$}} \\ \hline
		\endfoot
		
		\hline
		\endlastfoot
		
		\hline
		\multicolumn{2}{|l|}{\multirow{3}{*}{}}                           & \multirow{3}{*}{\textbf{Vosk}} & \multicolumn{4}{c|}{\textbf{DeepSpeech}}                                                                                   \\ \cline{4-7} 
		\multicolumn{2}{|l|}{}                                            &                                & \multicolumn{2}{c|}{Modelo Rhasspy}                                    & \multicolumn{2}{c|}{Modelo Polyglot}              \\ \cline{4-7} 
		\multicolumn{2}{|l|}{}                                            &   
		   & \multicolumn{1}{l|}{Scorer PocoLG} & \multicolumn{1}{l|}{Scorer KenLM} & \multicolumn{1}{l|}{Scorer PocoLG} & Scorer KenLM \\ \hline
		\multicolumn{1}{|l|}{\textbf{Texto Largo}}              &         & \makecell{CER = 0,1189 \\ WER = 0,2944}  & \multicolumn{1}{l|}{\makecell{$\alpha$ = 1,2/ $\beta$ = 1 \\ CER Máx/Med: \\ 0,7787/0,8064 \\ WER Máx / Med: \\ 0,8556/0,8678 }}  & \multicolumn{1}{l|}{\makecell{$\alpha$ = 1,3/ $\beta$ = 0,8 \\ CER Máx/Med: \\ 0,7787/0,7991 \\ WER Máx / Med: \\ 0,8600/0,8760 }} & \multicolumn{1}{l|}{\makecell{$\alpha$ = 1,2/ $\beta$ = 1 \\ CER Máx/Med: \\ 0,7787/0,8065 \\ WER Máx / Med: \\ 0,8556/0,8756 }}  & \makecell{$\alpha$ = 1,3/ $\beta$ = 0,8 \\ CER Máx/Med: \\ 0,7787/0,7990 \\ WER Máx / Med: \\ 0,8400/0,8680 } \\ \hline
		\multicolumn{1}{|l|}{\multirow{2}{*}{\textbf{Frase 1}}} & Arcadia &  \makecell{CER = 0,1304 \\ WER = 0,5}  & \multicolumn{1}{l|}{\makecell{$\alpha$ = 1,6/ $\beta$ = 0.7 \\ CER Máx/Med: \\0,0000/0,0855 \\ WER Máx / Med: \\ 0,0000/0,1070 }}              & \multicolumn{1}{l|}{\makecell{$\alpha$ = 1,9/ $\beta$ = 0,9 \\ CER Máx/Med: \\ 0,0000/0,0819 \\ WER Máx / Med: \\ 0,0000/0,1690 }}             & \multicolumn{1}{l|}{\makecell{$\alpha$ = 1,6/ $\beta$ = 0,7 \\ CER Máx/Med: \\ 0,0000/0,0819 \\ WER Máx / Med: \\ 0,0000/0,1070 }}              & \makecell{$\alpha$ = 1,9/ $\beta$ = 0,9 \\ CER Máx/Med: \\ 0,0000/0,0855 \\ WER Máx / Med: \\ 0,0000/0,1690 } \\ \cline{2-7} 
		\multicolumn{1}{|l|}{}                                  & Lumina  & \makecell{CER = 0,25 \\ WER = 0,045}  & \multicolumn{1}{l|}{\makecell{$\alpha$ = 1,9 / $\beta$ = 1\\ CER Mín./Med \\0,0909/0,1292\\ WER Mín./Med \\0,25/0,285}}              & \multicolumn{1}{l|}{\makecell{$\alpha$ = 1,9/$\beta$ = 1\\ CER Mín./Med \\0,0909/0,1359\\ WER Mín./Med \\0,25/0,291}}             & \multicolumn{1}{l|}{\makecell{$\alpha$ = 1,9 / $\beta$ = 1\\ CER Mín./Med \\0,0909/0,136\\ WER Mín./Med \\0,25/0,292}}              &  \makecell{$\alpha$ = 1,9 / $\beta$ = 1\\ CER Mín./Med \\0,0909/0,129\\ WER Mín./Med \\0,25/0,29} \\ \hline
		\multicolumn{1}{|l|}{\multirow{2}{*}{\textbf{Frase 2}}} & Arcadia & \makecell{CER = 0,1136 \\ WER = 0,375}   & \multicolumn{1}{l|}{\makecell{$\alpha$ = 1/ $\beta$ =1\\ CER Mín./Med \\0,1590/0,1590\\ WER Mín./Med \\0,5/0,7118}}              & \multicolumn{1}{l|}{\makecell{$\alpha$ = 1,4/ $\beta$ =1\\ CER Mín./Med \\0,2272/0,3966\\ WER Mín./Med \\0,625/0,7138}}             & \multicolumn{1}{l|}{\makecell{$\alpha$ = 1/ $\beta$ =1\\ CER Mín./Med \\0,2272/0,396\\ WER Mín./Med \\0,625/0,7138}}              &   \makecell{$\alpha$ = 1,4/ $\beta$ =1\\ CER Mín./Med \\0,1590/0,3865\\ WER Mín./Med \\0,5/0,7118}           \\ \cline{2-7} 
		\multicolumn{1}{|l|}{}                                  & Lumina  & \makecell{CER = 0,1395 \\ WER = 0,5}  & \multicolumn{1}{l|}{\makecell{$\alpha$ = 1/ $\beta$ =1\\ CER Mín./Med \\0,1162/0,2673\\ WER Mín./Med \\0,5/0,5155}}              & \multicolumn{1}{l|}{\makecell{$\alpha$ = 1/ $\beta$ =1\\ CER Mín./Med \\0,1162/0,3046\\ WER Mín./Med \\0,5/0,6188}}             & \multicolumn{1}{l|}{\makecell{$\alpha$ = 1/ $\beta$ =1\\ CER Mín./Med \\0,1162/0,305\\ WER Mín./Med \\0,5/0,619}}              &  \makecell{$\alpha$ = 1/ $\beta$ =1\\ CER Mín./Med \\0,1162/0,2673\\ WER Mín./Med \\0,5/0,5155} \\ \hline
		\multicolumn{1}{|l|}{\multirow{2}{*}{\textbf{Frase 3}}} & Arcadia & \makecell{CER = 0 \\ WER = 0} & \multicolumn{1}{l|}{\makecell{$\alpha$ = 1,8/ $\beta$ =1\\ CER Mín./Med \\0/0,13286\\ WER Mín./Med \\0/0,183}}              & \multicolumn{1}{l|}{\makecell{$\alpha$ = 1,8/ $\beta$ =1\\ CER Mín./Med \\0/0,1077\\ WER Mín./Med \\0/0,175}}             & \multicolumn{1}{l|}{\makecell{$\alpha$ = 1,8/ $\beta$ =1\\ CER Mín./Med \\0/0,1087\\ WER Mín./Med \\0/0,177}}              &  \makecell{$\alpha$ = 1,8/ $\beta$ =1\\ CER Mín./Med \\0/0,13286\\ WER Mín./Med \\0/0,183} \\ \cline{2-7} 
		\multicolumn{1}{|l|}{}                                  & Lumina  & \makecell{CER = 0,08 \\ WER = 0,2}  & \multicolumn{1}{l|}{\makecell{$\alpha$ = 1,1/ $\beta$ =1\\ CER Mín./Med \\0,12/0,37\\ WER Mín./Med \\0,4/0,668}}              & \multicolumn{1}{l|}{\makecell{$\alpha$ = 1,1/ $\beta$ =1\\ CER Mín./Med \\0,12/0,394\\ WER Mín./Med \\0,4/0,681}}             & \multicolumn{1}{l|}{\makecell{$\alpha$ = 1,1/ $\beta$ =1\\ CER Mín./Med \\0,12/0,396\\ WER Mín./Med \\0,4/0,683}}              & \makecell{$\alpha$ = 1,1/ $\beta$ =1\\ CER Mín./Med \\0,12/0,37\\ WER Mín./Med \\0,4/0,668} \\ \hline
		\multicolumn{1}{|l|}{\multirow{2}{*}{\textbf{Frase 4}}} & Arcadia & \makecell{CER = 0,75 \\ WER = 1}  & \multicolumn{1}{l|}{\makecell{$\alpha$ = 1,8/ $\beta$ =1\\ CER Mín./Med \\0/0,132\\ WER Mín./Med \\0/0,264}}              & \multicolumn{1}{l|}{\makecell{$\alpha$ = 1,7/ $\beta$ =1\\ CER Mín./Med \\0/0,136\\ WER Mín./Med \\0/0,273}}             & \multicolumn{1}{l|}{\makecell{$\alpha$ = 1,8/ $\beta$ =1\\ CER Mín./Med \\0/0,138\\ WER Mín./Med \\0/0,275}}              &  \makecell{$\alpha$ = 1,7/ $\beta$ =1\\ CER Mín./Med \\0/0,132\\ WER Mín./Med \\0/0,264}\\ \cline{2-7} 
		\multicolumn{1}{|l|}{}                                  & Lumina  & \makecell{CER = 0,1818 \\ WER = 0,5} & \multicolumn{1}{l|}{\makecell{$\alpha$ = 1,1/ $\beta$ =1\\ CER Mín./Med \\0,1818/0,4628\\ WER Mín./Med \\0,5/0,909}}              & \multicolumn{1}{l|}{\makecell{$\alpha$ = 1,1/ $\beta$ =1\\ CER Mín./Med \\0,1818/0,4628099\\ WER Mín./Med \\0,5/0,909}}             & \multicolumn{1}{l|}{\makecell{$\alpha$ = 1,1/ $\beta$ =1\\ CER Mín./Med \\0,1818/0,46515\\ WER Mín./Med \\0,5/0,913}}              & \makecell{$\alpha$ = 1,1/ $\beta$ =1\\ CER Mín./Med \\0,1818/0,4628\\ WER Mín./Med \\0,5/0,909} \\ \hline
		\multicolumn{1}{|l|}{\multirow{2}{*}{\textbf{Frase 5}}} & Arcadia & \makecell{CER = 0 \\ WER = 0} & \multicolumn{1}{l|}{\makecell{$\alpha$ = 1,9/ $\beta$ =0,7\\ CER Mín./Med \\0,24/0,268\\ WER Mín./Med \\0,5/0,533}}              & \multicolumn{1}{l|}{\makecell{$\alpha$ = 1,9/ $\beta$ =0,6\\ CER Mín./Med \\0,24/0,296\\ WER Mín./Med \\0,5/0,579}}             & \multicolumn{1}{l|}{\makecell{$\alpha$ = 1,9/ $\beta$ =0,7\\ CER Mín./Med \\0,24/0,297\\ WER Mín./Med \\0,5/0,579}}              & \makecell{$\alpha$ = 1,9/ $\beta$ =0,6\\ CER Mín./Med \\0,24/0,268\\ WER Mín./Med \\0,5/0,533} \\ \cline{2-7} 
		\multicolumn{1}{|l|}{}                                  & Lumina  & \makecell{CER = 0,083 \\ WER = 0,25} & \multicolumn{1}{l|}{\makecell{$\alpha$ = 1,6/ $\beta$ =1\\ CER Mín./Med \\0,3333/0,4621\\ WER Mín./Med \\0,75/0,857}}              & \multicolumn{1}{l|}{\makecell{$\alpha$ = 1,6/ $\beta$ =1\\ CER Mín./Med \\0,3333/0,4676\\ WER Mín./Med \\0,75/0,862}}             & \multicolumn{1}{l|}{\makecell{$\alpha$ = 1,6/ $\beta$ =1\\ CER Mín./Med \\0,3333/0,4687\\ WER Mín./Med \\0,75/0,86}}              & \makecell{$\alpha$ = 1,6/ $\beta$ =1\\ CER Mín./Med \\0,3333/0,4621\\ WER Mín./Med \\0,75/0,857} \\ \hline
		\multicolumn{1}{|l|}{\multirow{2}{*}{\textbf{Frase 6}}} & Arcadia & \makecell{CER = 0 \\ WER = 0} & \multicolumn{1}{l|}{\makecell{$\alpha$ = 1/ $\beta$ =1\\ CER Mín./Med \\0,022/0,0909\\ WER Mín./Med \\0,14286/0,175914}}              & \multicolumn{1}{l|}{\makecell{$\alpha$ = 1,9/ $\beta$ =1\\ CER Mín./Med \\0,066/0,0989\\ WER Mín./Med \\0,1428/0,1806}}             & \multicolumn{1}{l|}{\makecell{$\alpha$ = 1/ $\beta$ =1\\ CER Mín./Med \\0,066/0,0099\\ WER Mín./Med \\0,1429/0,181}}              &\makecell{$\alpha$ = 1,9/ $\beta$ =1\\ CER Mín./Med \\0,022/0,0909\\ WER Mín./Med \\0,1428/0,1759}\\ \cline{2-7} 
		\multicolumn{1}{|l|}{}                                  & Lumina  & \makecell{CER = 0,136 \\ WER = 0,5714} & \multicolumn{1}{l|}{\makecell{$\alpha$ = 1,4/ $\beta$ =1\\ CER Mín./Med \\0,0227/0,1590\\ WER Mín./Med \\0,1428/0,3116}}              & \multicolumn{1}{l|}{\makecell{$\alpha$ = 1,4/ $\beta$ =1\\ CER Mín./Med \\0,0227/0,1641\\ WER Mín./Med \\0,1428/0,3293}}             & \multicolumn{1}{l|}{\makecell{$\alpha$ = 1,4/ $\beta$ =1\\ CER Mín./Med \\0,0227/0,165\\ WER Mín./Med \\0,1428/0,3298}}              & \makecell{$\alpha$ = 1,4/ $\beta$ =1\\ CER Mín./Med \\0,0227/0,1590\\ WER Mín./Med \\0,1428/0,31168} \\ \hline
		\multicolumn{1}{|l|}{\multirow{2}{*}{\textbf{Frase 7}}} & Arcadia & \makecell{CER = 0,1071 \\ WER = 0,1666} & \multicolumn{1}{l|}{\makecell{$\alpha$ = 1,1/ $\beta$ =0,7\\ CER Mín./Med \\0,3392/0,4589\\ WER Mín./Med \\0,67/0,761}}              & \multicolumn{1}{l|}{\makecell{$\alpha$ = 1/ $\beta$ =1\\ CER Mín./Med \\0,3392/0,4579\\ WER Mín./Med \\0,75/0,7706}}             & \multicolumn{1}{l|}{\makecell{$\alpha$ = 1,1/ $\beta$ =0,7\\ CER Mín./Med \\0,3392/0,4589\\ WER Mín./Med \\0,75/0,7715}}              & \makecell{$\alpha$ = 1/ $\beta$ =1\\ CER Mín./Med \\0,3392/0,4589\\ WER Mín./Med \\0,667/0,7610} \\ \cline{2-7} 
		\multicolumn{1}{|l|}{}                                  & Lumina  & \makecell{CER = 0,1136 \\ WER = 0,1428} & \multicolumn{1}{l|}{\makecell{$\alpha$ = 1/ $\beta$ =0,9\\ CER Mín./Med \\0,4181/0,5359\\ WER Mín./Med \\0,75/0,86845}}              & \multicolumn{1}{l|}{\makecell{$\alpha$ = 1/ $\beta$ =0,9\\ CER Mín./Med \\0,4181/0,5316\\ WER Mín./Med \\0,75/0,8622}}             & \multicolumn{1}{l|}{\makecell{$\alpha$ = 1/ $\beta$ =0,9\\ CER Mín./Med \\0,4181/0,5326\\ WER Mín./Med \\0,75/0,8632}}              &  \makecell{$\alpha$ = 1/ $\beta$ =0,9\\ CER Mín./Med \\0,4181/0,5359\\ WER Mín./Med \\0,75/0,8684}   \\ \hline
		\multicolumn{1}{|l|}{\multirow{2}{*}{\textbf{Frase 8}}} & Arcadia & \makecell{CER = 0,1272 \\ WER = 0,25} & \multicolumn{1}{l|}{\makecell{$\alpha$ = 1,4/ $\beta$ =0,8\\ CER Mín./Med \\0,1136/0,1497\\ WER Mín./Med \\0,14286/0,36953}}              & \multicolumn{1}{l|}{\makecell{$\alpha$ = 1,4/ $\beta$ =1\\ CER Mín./Med \\0,1363/0,1487\\ WER Mín./Med \\0,2857/0,3636}}             & \multicolumn{1}{l|}{\makecell{$\alpha$ = 1,4/ $\beta$ =0,8\\ CER Mín./Med \\0,1363/0,1489\\ WER Mín./Med \\0,2857/0,3643}}              & \makecell{$\alpha$ = 1,4/ $\beta$ =1\\ CER Mín./Med \\0,1136/0,1496\\ WER Mín./Med \\0,1482/0,3695} \\ \cline{2-7} 
		\multicolumn{1}{|l|}{}                                  & Lumina  & \makecell{CER = 0,2093 \\ WER = 0,4285} & \multicolumn{1}{l|}{\makecell{$\alpha$ = 1,4/ $\beta$ =0,7\\ CER Mín./Med \\0,2093/0,3096\\ WER Mín./Med \\0,4285/0,5135}}              & \multicolumn{1}{l|}{\makecell{$\alpha$ = 1,4/ $\beta$ =0,9\\ CER Mín./Med \\0,1860/0,3117\\ WER Mín./Med \\0,4285/0,5088}}             & \multicolumn{1}{l|}{\makecell{$\alpha$ = 1,4/ $\beta$ =0,7\\ CER Mín./Med \\0,1860/0,3125\\ WER Mín./Med \\0,4286/0,5095}}              &  \makecell{$\alpha$ = 1,4/ $\beta$ =0,9\\ CER Mín./Med \\0,2093/0,3096\\ WER Mín./Med \\0,4285/0,5135} \\ \hline
	\end{xltabular}
}

También se han producido gráficas con los resultados, que comentaremos a continuación:

\begin{figure}[H]
	\includegraphics[width=\textwidth]{imagenes/WERCER_VoskIvan.png}
\end{figure}

\begin{table}
\begin{tabularx}{\textwidth}{|c|X|}
	\hline
	Texto & Predicción \\ \hline
	 Frase 1 (Arcadia) & arabia entiende la luz \\ \hline
	 Frase 1 (Lúmina) & lubina enciende la luz  \\ \hline
	 Frase 2 (Arcadia)& arcadia pone esa canción y tanto me gusto \\ \hline
	 Frase 2 (Lúmina)& nómina por mí esa canción que tanto me gustó \\ \hline
	 Frase 3 (Arcadia)& arcadia repite lo que diga \\ \hline
	 Frase 3 (Lúmina)& nómina repite lo que diga \\ \hline
	 Frase 4 (Arcadia)& porque había para \\ \hline
	 Frase 4 (Lúmina)& nómina para \\ \hline
	 Frase 5 (Arcadia)& arcadia espera un momento \\ \hline
	 Frase 5 (Lúmina)& nómina espera un momento \\ \hline
	 Frase 6 (Arcadia)& arcadia reproduce el informativo de la mañana \\ \hline
	 Frase 6 (Lúmina)& nómina reproducen informativos de la mañana \\ \hline
	 Frase 7 (Arcadia)& marcaría con una alarma a las ocho y media de la mañana \\ \hline
	 Frase 7 (Lúmina)& lubina con una norma a las ocho y media de la mañana \\ \hline
	 Frase 8 (Arcadia)& arcadia prepara un temporizador de cinco minutos \\ \hline
	 Frase 8 (Lúmina)& nómina preparado un temporizador de cinco minutos \\ \hline

\end{tabularx}
\end{table}

En el caso de Vosk, se aprecia que el Ratio de Error de Palabra es bastante difuso en las Frases cuando se usa Arcadia como \textit{trigger word} (de hecho, en una de las frases no ha acertado ni una palabra). Cuando vemos las frases donde se usa Lúmina como trigger word, el ratio es más alto pero menos difuso, encontrándose entre el 20 y casi el 60\%. En cuanto al texto más largo, ha captado una muy buena parte del mensaje. 
Si se tiene en cuenta que es un modelo de 55 MB una vez descomprimido, ofrece resultados bastante completos.

\begin{figure}[H]
	\includegraphics[width=\textwidth]{imagenes/MejoresResultados_DeepSpeechIvanPocoRhasspy.png} \\
	\includegraphics[width=0.5\textwidth]{imagenes/CER_DeepSpeechIvanPocoRhasspy.png} \hfill \includegraphics[width=0.5\textwidth]{imagenes/WER_DeepspeechIvanPocoRhasspy.png}
\end{figure}

Empezando en las combinaciones con DeepSpeech, donde se usa el Modelo Rhasspy y el Scorer más completo, podemos ver que los valores más óptimos de Beta se encuentran entre el 0.7 y el 1, pero los valores de Alfa son más dispares. 

Podemos apreciar también que la mayor disparidad entre lo que la red neuronal predice y lo que realmente se quería decir es del 85\% en el texto más largo. Las frases que comienzan en Arcadia parecen reconocerlas mejor que las que comienzan por Lúmina.

\begin{figure}[H]
	\includegraphics[width=\textwidth]{imagenes/MejoresResultados_DeepSpeechIvanKenRhasspy.png} \\
	\includegraphics[width=0.5\textwidth]{imagenes/CER_DeepSpeechIvanKenRhasspy.png} \hfill \includegraphics[width=0.5\textwidth]{imagenes/WER_DeepspeechIvanKenRhasspy.png}
\end{figure}

En esta combinación se usa el Modelo Rhasspy y el Scorer Ken, y podemos ver que los valores más óptimos de Beta se encuentran casi siempre en el 1, pero los valores de Alfa son bastante altos cuando se trata de reconocer una frase llamando a la primera opción como trigger word, generalmente entre 1,7 y 1,9. En cuanto a la segunda opción, los resultados o están muy bajos (cerca del 1,1) o altos (1,4 al 1,6 salvo la primera frase que usa un valor de 1,9).

La tónica se repite con esta combinación,donde el ratio de error en el texto largo es altísimo y reconoce muy bien el nombre de Arcadia frente al de Lúmina. En cuanto al ratio de error a nivel de caracteres, podemos ver que es muy similar al anterior.

\begin{figure}[H]
	\includegraphics[width=\textwidth]{imagenes/MejoresResultados_DeepSpeechIvanPocoPolyglot.png} \\
	\includegraphics[width=0.5\textwidth]{imagenes/CER_DeepSpeechIvanPocoPolyglot.png} \hfill \includegraphics[width=0.5\textwidth]{imagenes/WER_DeepspeechIvanPocoPolyglot.png}
\end{figure}

En este caso se usa el modelo Polyglot con el Scorer Poco. Los valores de Beta casi siempre están a 1, o al menos más allá de 0,7. En cuanto al Alfa, hay una tendencia a valores más altos en el caso de Arcadia, y a valores más bajos en el caso de Lúmina.

El valor del WER y del CER es muy parecido a los dos anteriores. Se sigue manteniendo la tendencia de que los textos largos tienen un valor alto. De hecho, si vemos esa opción, observamos que reconoce unos cuantos segundos y encima con muchos errores.

\begin{figure}[H]
	\includegraphics[width=\textwidth]{imagenes/MejoresResultados_DeepSpeechIvanKenPolyglot.png} \\
	\includegraphics[width=0.5\textwidth]{imagenes/CER_DeepSpeechIvanKenPolyglot.png} \hfill \includegraphics[width=0.5\textwidth]{imagenes/WER_DeepspeechIvanKenPolyglot.png}
\end{figure}

En la combinación del modelo Polyglot con el Scorer más liviano, los valores de Beta casi siempre están a 1, es decir, viene mejor cuanta más penalización se da por fallar. En cuanto al Alfa, hay una tendencia a valores más altos en el caso de Arcadia, y a valores más bajos en el caso de Lúmina, como en la anterior ocasión.

El valor del WER y del CER es muy parecido a los dos anteriores. Se sigue manteniendo la tendencia de que los textos largos tienen un valor alto, algo que vemos constante en los experimentos con DeepSpeech. De hecho, podemos ver en comparación lo que han deducido las 4 combinaciones de DeepSpeech y la otra API (véase tabla \ref{tab:predicts} en la página \pageref{tab:predicts})

\begin{table}
	\begin{tabularx}{\textwidth}{|c|X|}
		\hline
		\makecell{Vosk} & lo la y arcadia se encuentran en un portal a medio camino entre sus casas cerca del centro comercial ellas habían quedado para ver la nueva película que tanto proporcionaban por la tele y buscar un momento entre su agenda para ir a verla comparte un beso auriculares con su amiga y entienden entre spotify y reproduce el tema de que perdí mientras pasión hacia su destino al llegar buscar el cine y la entrada de las palomitas para la sala un temporizador de cinco minutos en la pantalla que además no paraba de poner publicidad de repente todo separa por un corte de luz toda esta oscura si alguien entiende la luz de su teléfono aunque no sirve de mucho hasta que vuelva suministro funcionar y ya puedan ver la película al terminar ya que tiempo hace para saber si esperar a que recojan o volver andando aunque lo piensa mejor tras mirar qué hora es y venir tomar un café antes de irse \\ \hline
		\makecell{DeepSpeech,\\ Modelo Rhasspy,\\ Scorer Poco} & una arcadia se encuentran en un portal a medio camino entre sus casas cerca del centro comercial ella habian quedado para ver la nueva pelicula que tanto proporcionada por la tele y buscar numenoreanos eeaaoaeecaoeee \\ \hline
		\makecell{DeepSpeech,\\ Modelo Rhasspy,\\ Scorer Ken} & una arcadia se encuentran en un portal a medio camino entre sus casas cerca del centro comercial ella habian quedado para ver la nueva pelicula que tanto proporcionada por la tele y buscar numenoreanos eeaaoaeecaoeee \\ \hline
		\makecell{DeepSpeech,\\ Modelo Polyglot,\\ Scorer Poco} & una arcadia se encuentran en un portal a medio camino entre sus casas cerca del centro comercial ella habian quedado para ver la nueva pelicula que tanto proporcionada por la tele y buscar numenoreanos eeaaoaeecaoeee \\ \hline
		\makecell{DeepSpeech,\\ Modelo Polyglot,\\ Scorer Ken} & una arcadia se encuentran en un portal a medio camino entre sus casas cerca del centro comercial ella habian quedado para ver la nueva pelicula que tanto proporcionada por la tele y buscar numenoreanos eeaaoaeecaoeee\\ \hline
		
	\end{tabularx}
	\caption{Textos que arrojan las APIs al analizar el audio más largo.}
	\label{tab:predicts}
\end{table}

Si bien hemos hablado de los datos en su propio contexto, ¿qué pasa si comparamos los resultados entre sí? Para ello se ha exportado las respuestas finales en gráficas para comparar su rendimiento ante las mismas pruebas y sacar algunas conclusiones.

\begin{figure}[H]
	\includegraphics[width=0.5\textwidth]{imagenes/Alfas.png} \hfill \includegraphics[width=0.5\textwidth]{imagenes/Betas.png}
\end{figure}

En el caso de DeepSpeech, los ajustes de Alfa y Beta son iguales en el caso de los modelos, por lo que usar uno u otro no nos daría ninguna ventaja. Por otro lado, los scorers sí alteran los parámetros, observando que en el Scorer Poco se da el mismo o menos peso al modelo del lenguaje en general que en Scorer Ken (como excepción podemos ver en la Frase 7 con Arcadia que el peso es menor). También se le da menos castigo en el predictor más completo (salvo en el Texto Largo y la Variante con Arcadia de la Frase 5).

\begin{figure}[H]
	\includegraphics[width=0.5\textwidth]{imagenes/CERMedios.png} \hfill \includegraphics[width=0.5\textwidth]{imagenes/WERMedios.png}
\end{figure}

Por el Character Error Rate, vemos que al comparar letra a letra Vosk mantiene unos niveles bastante bajos salvo por su pico del 75\%. En cuanto al Ratio de Error a nivel de palabra, la cifra sube a valores de hasta un 50\% salvo en la misma frase del pico anterior, donde falla toda la predicción.
En el caso de DeepSpeech, hay muy pocas diferencias en sus valores medios entre sus 4 variantes. Como hemos visto anteriormente, debido a su comportamiento en el texto largo donde reconoce unos cuantos segundos, las relaciones de errores se disparan al tope del 80
\% en el caso del CER y del casi 90\% en el caso del WER

\begin{figure}[H]
	\includegraphics[width=0.5\textwidth]{imagenes/CERMejores.png} \hfill \includegraphics[width=0.5\textwidth]{imagenes/WERMejores.png}
\end{figure}

En cuanto a los resultados arrojados en las mejores generaciones, vemos que en el frente de DeepSpeech sigue siendo bastante homogéneo, si bien la combinación del Modelo Polyglot con el Scorer Ken es algo mejor en algunas ocasiones. Sin embargo, hay muchos casos donde Vosk es mejor que las 4 combinaciones, aunque podemos ver puntos aislados sorprendentes, como ver que DeepSpeech ha sido capaz de descifrar perfectamente qué se estaba hablando, cuando la alternativa falló completamente.

Otra conclusión que podemos sacar son, por ejemplo, que de los nombres propuestos, ``Arcadia'' se reconoce más veces que ``Lúmina'' (de hecho, esta última se ha reconocido sólo una vez por DeepSpeech), aunque a veces se confundan con otras palabras (como \textit{arabia}, \textit{marcaría}, o \textit{nómina}). Para permitir que se entendieran perfectamente estas palabras habría que entrenar los modelos para que las aceptaran e integraran en su vocabulario. Eso sí, hacerlo requeriría varias tarjetas gráficas y un tiempo de procesamiento de alrededor de 100 horas para hacer una sintonía fina. Otra opción sería poner alguna palabra parecida como \textit{trigger word} también, pero eso haría activarse el asistente en alguna ocasión inesperada según si la palabra es más o menos común. 

Por lo visto sobre los nombres, en este estado podemos darle nombre a nuestro Asistente en función del comportamiento de las opciones cuando se menciona esta palabra. También en este punto podemos elegir la API que menos fallo nos daría. Para ello, vamos a comprobar su promedio y desviación típica.

\begin{xltabular}{\textwidth}{|X|X|X|X|X|}
	\hline
	Método & CER Promedio & Desv. Típica de CER & WER Promedio & Desv. Típica de WER \\ \hline
	DeepSpeech + Rhasspy + Ken (Mínimo/Medio) & 0,1916/0,3115 & 0,1968/0,1966 & 0,4047/0,5349 & 0,2855/0,2763 \\ \hline
	DeepSpeech + Rhasspy + Poco (Mínimo/Medio) & 0,1850/0,2929 & 0,1993/0,1990 & 0,3845/0,5277 & 0,2814/0,2705 \\ \hline
	DeepSpeech + Polyglot + Ken (Mínimo/Medio) & 0,1850/0,3063 & 0,1993/0,1971 & 0,3844/0,5280 & 0,2805/0,2702 \\ \hline
	DeepSpeech + Polyglot + Poco (Mínimo/Medio) & 0,1916/0,3070 & 0,1968/0,2038 & 0,4038/0,5357 & 0,2840/0,2763 \\ \hline
	Vosk & 0,1375 & 0,1689 & 0,3194 & 0,2537 \\ \hline
\end{xltabular} 

Por los datos arrojados, la API que más nos convendría usar es Vosk, ya que tiende a acertar más lo que se quiere decir. 

También tenemos un nombre para nuestro asistente, pudiendo añadir algo de personalidad al producto resultante. \textit{Así que, con todos ustedes, ¡os presento a Arcadia!}

\section{Eligiendo una API de Síntesis de Voz}
Por la otra parte, sintetizar la voz nos permite ofrecer el feedback al usuario los resultados a lo que se piden. Al igual que en el Reconocimiento del Habla, las grandes compañías tienen su propia implementación en un formato "Software como servicio" (como Google TTS o IBM Watson TTS).

En este campo tenemos también otro par de APIs que usan sistemas de Código Abierto. Estos son:

\begin{itemize}
	\item \textbf{Festival TTS}: Creada por la Universidad de Edimburgo, requiere de una versión especial del programa para poder conectarse con la API en Python y así dictar lo que se ha escrito.
	\item \textbf{eSpeak}:  
\end{itemize}

\section{Diseñando el problema}
 


	% Desarrollo bajo sprints: 
	% 	1. Permitir registros y login de usuarios
	% 	2. Desarrollo del sistema de incidencias
	% 	3. Desarrollo del sistema de denuncias administrativas y accidentes
	% 	4. Desarrollo del sistema de croquis
	%   5. Instalación de la aplicación de manera automática
	\newpage 
	\chapter{Implementación}

\noindent\fbox{
	\parbox{\textwidth}{
		Tras lo previsto en el anterior capítulo, aquí se comentará cómo se ha realizado el sistema resultante, en base a la planificación y los diseños, de forma que se acabe con, al menos, un Producto Mínimo Viable.
	}
}
\newline

La implementación del software se ha dividido en los anteriores sprints. Estos, han sido definidos en Github
y cada uno de ellos contiene un grupo de \textit{issues} que se corresponden con las distintas
mejoras que se han ido incorporando al software a lo largo de su desarrollo.\\

\section{Sprint 1}
\subsection{Implementación de la reproducción de audio}
Para esta parte se hizo un adaptador genérico con el único método que necesitaríamos
de los reproductores: Reproducir un archivo con el método \texttt{play()}

Tras ello se implementaron dos versiones del reproductor con base en ese adaptador,
usando PyAudio \cite{pyaudio} y FFMpeg \cite{ffmpeg}. En ambas plataformas se pudo implementar fácilmente, aunque PyAudio es más sencillo de integrar pero enfocado en archivos \texttt{.wav}, mientras que para usar FFMpeg había que usar el comando \texttt{ffplay} con las opciones, aunque este nos permite usar cualquier archivo de audio, incluso streamings de audio, como puede ser la radio por Internet.

\subsection{Implementación de la grabación de audio}
Para esta parte se hizo un adaptador genérico con el método que necesitaríamos
de los reproductores: Reproducir un archivo con el método \\ \texttt{record\_audio()}. De cara al futuro, también deberíamos permitir encauzar los datos que salen del micrófono para tratarlos posteriormente con el método \texttt{stream\_audio()}

En este caso se ha querido realizar algo similar con respecto al punto anterior, pero la grabación del audio desde FFMpeg daba una señal con muchas interferencias, debido a que se debe apuntar a una interfaz hardware concreta, a diferencia de PyAudio, que usa el dispositivo por defecto. Así, al coger el audio del micrófono interno, se generaba mucho ruido de lo que captaba alrededor de este.

\subsection{Adaptación de la API de Speech Recognition}
Tal como se discutió en el anterior capítulo, se acabó eligiendo Vosk\cite{vosk}, el cual tiene una librería en Python a través de pip que nos permitía usarla.
Para permitir el desarrollo de futuros adaptadores para otros sistemas de Reconocimiento del Habla, se ha optado por hacer otra Interfaz que sirva de marco para definir los métodos que se usarán en nuestro sistema.
De esta interfaz se implementará la clase \texttt{VoskAdapter} del cual podemos sacar la transcripción de un audio.

\subsection{Implementación del streaming de audio}
A la hora de hacer las pruebas surgía una inquietud. Habría que perder un tiempo en grabar y pasarlo al sistema de reconocimiento del habla, de un audio que luego se borraría y puede ocupar espacio en memoria. Si realmente lo único que necesitamos es la transcripción, ¿podemos encauzar directamente la grabación del audio con el Speech Recognizer para obtenerlo? 

Para ello había que implicar tanto el Grabador de Audio como el Software que nos transcriba el audio. Para ello haremos uso en la interfaz de \\ \texttt{GenericAudioRecorder} del método \texttt{stream\_audio()}, que nos devolvería la referencia a los datos del audio en directo, para poder usarlos luego en otro lado.

\subsection{Conectando el streaming con el Speech Recognition}
Por otra parte, nuestro adaptador de SR nos debería permitir coger esos datos e ir procesando esa transcripción conforme va leyéndolos.

Esta parte se ha conseguido realizar con PyAudio y Vosk, siguiendo esta aproximación:

\begin{enumerate}
	\item Abrimos el streaming de audio y lo pasamos como referencia
	\item Mientras que no esté en silencio un número S de bloques (que podemos parametrizar):
	\begin{enumerate}
		\item Cogemos un bloque de N bytes (N se puede parametrizar) para leer.
		\item Comprobamos que no esté en silencio. Si está en silencio (es decir, está en un volumen más bajo de un límite que podemos parametrizar), sumará el contador de S
		\item Si no lo está, añadirá a la cadena de texto la transcripción de ese bloque
	\end{enumerate}
	\item Una vez se salga del bucle, cerramos el stream y devolvemos la cadena de la transcripción.
\end{enumerate}

En FFMpeg, debido a los problemas con la grabación, se ha optado por no hacer tampoco el streaming. Además, habría un problema con respecto al encauzamiento, puesto que espera un subproceso en vez de más código, por lo que sería más complicado de trabajar.

\subsection{Adaptación del API de Text-to-Speech}
Como lo que queremos es generar la voz, en nuestra Interfaz para adaptadores crearemos un método para generar la voz. Esta voz se podría reproducir directamente o guardarse para después ser reproducido. En este caso, aplicaremos la segunda opción por si se necesitara repetir lo último que se ha dicho.

Para el Text-to-Speech hay que usar un repositorio externo llamado NanoTTS \cite{nanotts}, tal como comentamos en el apartado del Análisis. sólo crearemos un método llamado \texttt{generate\_voice()}.

Para poder usar este programa, hay que descargarlo del repositorio y ejecutar su \texttt{Makefile} (aunque todo está bien explicado en el README del proyecto en GitHub \cite{nanotts}).

A diferencia de los programas instalados por \textit{Aptitude}, hay que exportar el \texttt{PATH} de este Text-to-Speech, cosa que deberemos hacer cada vez que vayamos a usarlo (o integrarlo como parte de nuestra configuración de \textit{Bash}).

La librería que nos permite implementar NanoTTS en Arcadia también hace otra suposición sobre el sitio donde están los archivos para los idiomas (según el código, deberían estar en \texttt{/usr/bin/lang/}), por lo que hay que copiar esos archivos en el lugar.

Una vez cumplimentados los requisitos, sólo habría que poner un sitio donde guardar el audio y usar el método \texttt{speaks()} que generará el audio y lo guardará donde lo indiquemos. Además, a la hora de crear la instancia, nos permite parametrizar la velocidad de lectura y el tono.

Nos da como resultado un archivo \texttt{.wav} que podemos reproducir con los adaptadores que hemos hecho anteriormente.


\section{Sprint 2}
\subsection{Creación del bot en RASA}
Usar Rasa \cite{rasa} como un módulo de Python es bastante cómodo ya que puede comportarse como un CLI, facilitando gran parte de la creación de chatbots.

Además, nos dan un proyecto de prueba donde explican cómo funcionan los archivos y cómo entrenar nuestro modelo, usando el comando \texttt{rasa init} o \texttt{python -m rasa init}.

Dentro de la carpeta que nos crea, nos encontramos con el siguiente directorio:


\dirtree{%
	.1 chatbot.
	.2 .rasa.
	.3 cache.
	.4 (Incluye todos los archivos temporales de entrenamiento de Rasa).
	.2 actions.
	.3 actions.py.
	.3 \_\_init\_\_.py.
	.2 data.
	.3 nlu.yml.
	.3 rules.yml.
	.3 stories.yml.
	.2 models.
	.3 (Incluye los modelos en formato .tar.gz).
	.2 tests.
	.3 test\_stories.yml.
	.2 config.yml.
	.2 credentials.yml.
	.2 domain.yml.
	.2 endpoints.yml.
}

Rasa tiene como unidades básicas para entender la pregunta las intenciones (o \textit{intents}), que son las ideas que se quieren transmitir. Así, cuando alguien saluda, lo puede decir de varias formas (\textit{Hola, Buenos días, Hey} ...). Podremos controlar los ejemplos de las intenciones (y crear las nuestras propias) en \texttt{data/nlu.yml}

Para dar las respuestas, Rasa usa dos conceptos \cite{rasa-respuestas}:
\begin{itemize}
	\item \textbf{Declaraciones (o \textit{utterances}):} Son aquellas respuestas que sólo precisan de una serie de textos (Por ejemplo, si pedimos que salude, podemos decir que siempre responda con un \textit{Muy buenas, ¿qué tal?}). Estas respuestas se declaran en la sección \texttt{responses} de \texttt{domain.yml}.
	\item \textbf{Acciones (o \textit{actions}):} Son aquellas respuestas más complejas que necesitan de información dinámica (como poner frases aleatorias o decir la hora). Estas respuestas se declaran en la sección \texttt{actions} de \texttt{domain.yml}, y se deben programar en \texttt{actions/actions.py} (se explicará cómo hacerlo en secciones posteriores).
\end{itemize}

Para poder saber cómo relacionar las preguntas y las respuestas, tenemos dos maneras. Por una parte, podemos hacer que sigan reglas \cite{rasa-rules} de forma que cada vez que se manifieste una intención, debemos siempre relacionarlo con una respuesta. Estas reglas se pueden establecer en el archivo \texttt{data/rules.yml}

Por otra parte, si queremos que sigan un flujo de conversación, podemos hacer historias de usuario \cite{rasa-stories}, donde representamos una serie de intenciones y sus respuestas en forma de acción o declaración, que podrán usarse para entrenar el modelo a base de ejemplos. Estas historias de usuario podemos tenerlas como parte del entrenamiento o como parte de la validación (que quedan aparte del entrenamiento para comprobar que lo aprendido se puede exportar a otros casos parecidos). Los casos de uso de entrenamiento se podrán introducir en \texttt{data/stories.yml}, y los de validación, en \texttt{tests/test\_stories.yml}

Cuando se hagan cambios en el modelo, hay que volverlo a entrenar. Desde el archivo de configuración (\texttt{config.yml}) podemos hacer ajustes en las políticas de entrenamiento.
Para entrenar desde Rasa se ejecuta \texttt{rasa train}, quien usando los casos de uso de entrenamiento como de validación, podrá aprender a entender las intenciones y dar así una respuesta.
Para poner en marcha nuestro bot podemos ejecutar \texttt{rasa run}, iniciando así el servidor con la instancia.



\subsection{Conexión del chatbot con scripts externos de Python}
El servidor del chatbot se puede comunicar a través de una petición HTTP a localhost con un puerto que podemos especificar, de forma que podríamos enviar las peticiones y recibir sus respuestas, que podemos tratar posteriormente (por ejemplo, se pueden leer usando Text-to-Speech).

Para ello, desde el archivo \texttt{credentials.yml} se puede activar la opción \texttt{rest} para dejar el puerto abierto y poder hacer peticiones en otro lado.

Por otro lado, en Python, podemos usar la librería \texttt{requests} para hacer una petición POST en formato JSON en el que pasamos como entrada la siguiente estructura:

\lstset{frame=tb,
	language=Java,
	aboveskip=3mm,
	belowskip=3mm,
	showstringspaces=false,
	columns=flexible,
	commentstyle=\color{green},
	stringstyle=\color{blue},
	keywordstyle=\color{magenta},
	basicstyle={\small\ttfamily},
	numbers=none,
	breaklines=true,
	breakatwhitespace=true,
	tabsize=3
}

\begin{lstlisting}
	{
		"sender": /*Nombre o ID del emisor*/,
		"message": /*El texto a procesar*/
	}
\end{lstlisting}

Como resultado nos devuelve otro JSON con esta otra estructura:

\begin{lstlisting}
	[ /*Lista de JSON de la misma estructura, para cada respuesta que se deba devolver*/
	{
	 "recipient_id": /*ID o nombre del receptor*/,
	 "text": /*Texto con la respuesta*/
 	}, 
	/*...*/
	]
\end{lstlisting}

Alternativamente, se pueden devolver otras etiquetas como \textit{image} o \textit{custom}, donde podemos personalizar la respuesta.

De aquí, lo que necesitaríamos es el texto o referencia de cada respuesta (indicado como \texttt{text}), lo cual podemos recopilar en un string para posteriormente devolverlo, o el campo custom que podemos tratar como sea necesario.

\subsection{Implementando el flujo básico del chatbot}
Una vez que se ha logrado hacer la conexión, tendríamos que implementar el flujo de nuestro Asistente, como hemos visto en la Figura \ref{fig:diagramaflujo}. Para ello, nuestra primera versión del programa realiza estos pasos en el bucle principal:
\begin{enumerate}
	\item Hace el reconocimiento del habla en streaming para obtener su transcripción:
	\item Busca el \textit{trigger word} en la cadena y se queda con lo que hay después si es así. Si no, vuelve al paso 1.
	\item Envía la petición de esa transcripción recortada al chatbot para que nos devuelva su respuesta
	\item Genera la voz con la respuesta
	\item Reproduce la voz con la respuesta. 
\end{enumerate}



\subsection{Controlando algunos fallos de conexión}
Es posible que en algún momento pueda fallar la conexión entre Rasa y nuestro asistente, así que tendremos que tenerlo en cuenta. Una opción que podemos implementar es capturar las excepciones que digan que no puedan establecer la conexión (como \texttt{requests.exceptions.ConnectionError}), y hacer que nos diga que no puede conectarse con la Base de Conocimientos.

\section{Sprint 3}
\subsection{Haciendo funciones con solo texto}
Este tipo de funciones son bastante simples ya que sólo hay que crear una intención, una declaración y unir ambas partes usando una regla o historia.

Como ejemplo, podríamos hacer que si preguntamos sobre el Asistente nos hable un poquito de él. Para ello:

\begin{enumerate}
	\item Vamos primero a declarar una intención en \texttt{data/nlu.yml} llamado \\ \texttt{about\_me} donde podemos poner de ejemplos algunas frases como \textit{Háblame de ti} o \textit{¿Cómo te describirías?}:
	\begin{lstlisting}
		- intent: talk_about_me
		examples: |
		- hablame sobre ti
		- hablame de ti
		- que me puedes decir sobre ti
		- que eres
		- cuentame sobre ti
		- como te describirias
		- y tu quien eres
	\end{lstlisting}

	\item También tendríamos que darle una declaración en \texttt{domain.yml}
	\begin{lstlisting}
		responses:
		utter_about_me:
		- text: Pues soy un Asistente Virtual como los demas, pero aunque no se mucho, voy aprendiendo cosas poco a poco. Puede que sea la becaria del grupo
	\end{lstlisting}

	\item Tenemos que declarar las intenciones también en \texttt{domain.yml}
	\begin{lstlisting}
		intents:
		- affirm
		- deny
		- goodbye
		- greet
		- mood_great
		- mood_unhappy
		- nlu_fallback
		- talk_about_me
	\end{lstlisting}

	\item Finalmente , hay que poner a entrenar el modelo y probarlo usando \texttt{rasa train}
\end{enumerate}

\subsection{¿Y si preguntan algo que no sabe?}

En la documentación de Rasa se explica que hay una función de fallback \cite{rasa-fallback} donde caerían las respuestas que no tienen parecido con ninguna. En ese caso, nos dirige a \texttt{nlu\_fallback}, donde podemos definir como \textit{utterance} una respuesta por defecto.

Esto nos genera un problema debido a que no está entrenado para estos casos, aunque podemos entrenarlo de forma interactiva usando \texttt{rasa \\ interactive}, donde podremos escribir nuestras peticiones, controlar cuál intención debe lanzarse, y qué respuesta debería devolverse. 

De esta forma, al poner cualquier otra frase al que no se le pueda asociar ninguna acción (por ejemplo, si decimos \textit{¿Cuál es la raíz cuadrada de una sardina?}, una frase que no tiene mucho sentido, o si no se van a implementar funciones de cálculo, no tendría respuesta), se le puede aplicar una regla que diga que si la intención es el \textit{fallback}, nos llame a la declaración que hemos mencionado antes.

\subsection{Programando funciones más complejas con Rasa Actions}
Lo que hemos comentado en los puntos anteriores sólo sirven para dar respuestas fijas, de forma que siempre se dirá lo mismo. ¿Qué pasaría entonces si necesitamos hacer acciones dinámicas, como dar la hora? En esencia, sólo cambiaríamos como hemos dicho anteriormente de devolver una declaración a a llamar a una acción. ¿Cómo se haría?

\begin{enumerate}
	\item Vamos primero a declarar una intención en \texttt{data/nlu.yml} llamado \\ \texttt{tell\_time} donde podemos poner de ejemplos algunas frases como \textit{¿Qué hora es?} o \textit{Dime la hora}:
	\begin{lstlisting}
		- intent: tell_time
		examples: |
		- que hora es
		- dime la hora
		- me podrias decir la hora
		- a que hora estamos
	\end{lstlisting}
	
	\item Tenemos que declarar las intenciones en \texttt{domain.yml}
	\begin{lstlisting}
		intents:
		- affirm
		- deny
		- goodbye
		- greet
		- mood_great
		- mood_unhappy
		- nlu_fallback
		- talk_about_me
		- tell_time
	\end{lstlisting}

	\item A diferencia de las declaraciones, ahora debemos programar nuestras acciones en el fichero \texttt{actions/actions.py}. Esta sería la estructura de la clase que implementaría la acción, con los respectivos comentarios:

	\begin{lstlisting}[language=Python]
		class ActionTellTime(Action):
		
		def name(self) -> Text:
		return "action_tell_time"
		
		def run(self, dispatcher: CollectingDispatcher,
		tracker: Tracker,
		domain: Dict[Text, Any]) -> List[Dict[Text, Any]]:
		
		dispatcher.utter_message(text=u"Son las {0}".format(datetime.datetime.now().strftime('%H horas y %M minutos')))
		
		return []
	\end{lstlisting}

	\item Debemos listar también la acción en \texttt{domain.yml}, en la sección \texttt{actions}:
	\begin{lstlisting}
		actions:
		- action_tell_time
	\end{lstlisting}
	
	\item Finalmente , hay que poner a entrenar el modelo y probarlo usando \texttt{rasa train}
\end{enumerate}

\subsection{Conectarse a otros lugares de Internet}
\label{subsection:7-3-4}
Al igual que para conectarnos desde Arcadia hasta la fuente de conocimientos de Rasa, podemos también usar la librería \texttt{requests} para poder conectarnos a una API que nos dé la información. De esta forma, y decodificando el JSON que nos den de resultado, podemos conectarnos a Internet, sacar el JSON, y poder tratar la información de este en nuestro caso.

Como ejemplo en esta parte, podríamos consultar el precio de la luz usando la API de preciodelaluz.com, donde podemos ver los valores máximos, mínimos y medios de su valor en el día, además de ofrecernos el precio para esa hora.

Por ello, usando los endpoints que nos ofrecen, con la siguiente estructura (que según el endpoint tiene más o menos):
\begin{lstlisting}
	{
		"date": "27-05-2022",
		"hour": "19-20",
		"is-cheap": false,
		"is-under-avg": false,
		"market": "PVPC",
		"price": 289.87,
		"units": "EUR/Mwh"
	}
\end{lstlisting}

Podemos coger los valores de \texttt{hour} y \texttt{price} para crear una frase que el sistema pueda leer.
\begin{lstlisting}[language=Python]
	class ActionLightPrice(Action):
	def name(self) -> Text:
	return "action_light_price"
	
	def run(self, dispatcher: CollectingDispatcher,
	tracker: Tracker,
	domain: Dict[Text, Any]) -> List[Dict[Text, Any]]:
	
	try:
	request = requests.get('https://api.preciodelaluz.org/v1/prices/now?zone=PCB')
	if request is not None:
	price_now = u'{0} euros el megavatio hora'.format(request.json()['price'])
	# Lo mismo para obtener el precio medio (avg_price), el maximo (max_price) y el minimo (min_price)
	
	dispatcher.utter_message(text=u"En Espana, a esta hora, el precio es de {0}. De media, hoy pagaremos {1}, siendo la hora mas barata {2}; y la mas cara {3}".format(price_now,avg_price,min_price,max_price))
	except requests.exceptions.ConnectionError as e:
	dispatcher.utter_message(ActionsSettings.NO_CONNECTION_TO_INTERNET)
	
	return []
\end{lstlisting}

También podríamos aprovechar que tenemos conexión a Internet para tener audios que se pueden reproducir en el Asistente, incluso fuentes de audio en directo como las radios en Internet.

En este caso, el sistema es más sencillo, ya que como se ha discutido en la sección ... podemos usar valores personalizados que nos permitan poner URLs que podamos tratar de forma independiente en el flujo.

Para ello podemos hacer uso de campos \textit{custom} que nos permitan poner valores personalizados, y hacer un parsing de esos bloques (aunque en este caso solo ha sido probado con una emisora):

\begin{lstlisting}
	utter_play_radio:
		- custom:
			text: De acuerdo, te pongo /*Nombre de emisora*/.
			audio: /*URL*/
\end{lstlisting}

\subsection{Funciones contextuales}

Un último caso que sería interesante de analizar sería poder tratar las frases en un contexto, del que sacar un valor que trataremos para sacar otra información. Para ello, un caso de uso que pueda aplicar Arcadia en este sentido podría ser preguntar por el tiempo en una ciudad.

Podemos sacar provecho en las funcionalidades de Rasa de las entidades, una unidad dentro de la frase que varía el sentido de lo que se ha dicho parcialmente y que se debe tener en cuenta para procesar la respuesta.

Por ello, no es lo mismo preguntar por el tiempo en Granada que preguntarlo en Barcelona, teniendo en cuenta las diferencias entre dónde están situadas, si están pasando por fenómenos en ese momento que no cubren ambas zonas, y otras condiciones meteorológicas. Por tanto, las respuestas van a cambiar, pero lo único que cambiará al preguntar será la ciudad donde preguntemos.

Para poder extraer esa entidad, usando el ejemplo del tiempo en esa ciudad, podemos seguir los siguientes pasos:
\begin{enumerate}
	\item Declarar la entidad en el archivo \texttt{domain.yml}:
	\begin{lstlisting}
		entities:
			- city /*Nombre de la entidad. Es la ciudad*/
		slots:
			city:
				type: text
				influence_conversation: true /*Esta variable indica si este campo es importante para entender lo que se quiere pedir*/
				mappings:
				- type: custom
	\end{lstlisting}
	\item Poner unos ejemplos en el archivo \texttt{data/nlu.yml}:
	\begin{lstlisting}
		- intent: city_weather
		examples: |
		- que tiempo hace en [granada](city)
		- como esta el tiempo en [Malaga](city)
		- como esta el clima en [madrid](city)
		- como esta [Londres](city) hoy
		- dime que tiempo hace en [paris](city)
	\end{lstlisting}
	\item Declarar una historia de usuario con algunos ejemplos de la entidad (o sea, ciudades) en los archivos  \texttt{data/stories.yml} y  \texttt{test/test\_stories.yml}:
	\begin{lstlisting}
		- story: interactive_story_2
		steps:
		- intent: greet
		- action: utter_greet
		
		- intent: city_weather
		entities:
		- city: granada /*En este tipo de historias debemos indicar la entidad relacionada con la intencion. Como en este caso exigimos que la ciudad aparezca, en el ejemplo ponemos una*/
		- action: action_tell_weather
		
		- intent: city_weather
		entities:
		- city: sevilla
		- action: action_tell_weather
		
	\end{lstlisting}

	Para que sirva de ayuda a la hora de crear historias se puede emplear el comando \texttt{rasa interactive}, de forma que podemos probar en una \textit{shell} si las intenciones y acciones se ejecutan correctamente, si se extraen bien las entidades, y de no ser así, podemos cambiar la ejecución. Al terminar una sesión interactiva, se puede guardar estos tests en los archivos de Rasa que se requieran.
	
	\item Entrenamos el modelo usando \texttt{rasa train}.
\end{enumerate}

\subsection{Últimos ajustes}

Un cambio de última hora, debido a la posibilidad de poner audios usando lo que se ha comentado en la sección \ref{subsection:7-3-4} , necesitamos hacer un pequeño cambio en el flujo básico para tratar ese caso, y de ser así,  activar el reproductor con la URL.

Además se ha pasado el proyecto por el linter \textit{PyLint} \cite{pylint}. Un linter nos permite ver aquellos fallos lógicos y/o de estilo en conformidad con las reglas del estándar PEP-8 \cite{pep8}. Si bien se han podido corregir muchas de ellas, hay un par de advertencias que he decidido saltar en la evaluación del código del proyecto:
\begin{itemize}
	\item \textbf{R0903: \textit{Too few public methods (x/2)}} Esta advertencia viene debida a que hay clases que apenas tienen un sólo método, pero para mantener cierta integridad con respecto a lo descrito en el capítulo anterior, se ha preferido continuar con la arquitectura de clases.
	\item \textbf{W0236: \textit{Method \texttt{r} was expected to be \texttt{r}, found it instead as \texttt{r'}}} Al parecer, esta advertencia sucedía en Rasa debido a que el método \texttt{run()} suele estar asociado a funciones asíncronas, si bien esto está estandarizado en el sistema de Rasa. Por tanto se ha decidido ignorar este tipo de advertencia.
\end{itemize}

\begin{table}[H]
	\centering
	\begin{tabularx}{\textwidth}{|>{\columncolor{mintgreen}}c>{\columncolor{mintgreen}}X|}
		\hline
		\includegraphics[width=30pt]{imagenes/Tarea_completada.png} & Con ello, cumplimos los Objetivos \textbf{O-DD 5.} (Con base a la arquitectura, codificar lo necesario para la implementación de la aplicación.) \\
		\hline
	\end{tabularx}
\end{table}

\section{Sprint 4}
\subsection{Terminando la documentación del proyecto}
Para realizar la documentación se ha hecho uso de \textit{PyDoc}, compilando toda la información que se ha escrito en los métodos en varios archivos web. Si bien puede servir como punto de inicio, lo ideal sería pasarlo en el formato de Markdown.

Además se ha descrito la manera de ejecutar el programa y probarlo.

También se ha decidido, de cara a la futura muestra del código a los repositorios públicos, añadir un archivo de Código de Conducta y un archivo de Contribuciones.

\begin{itemize}
	\item Un \textbf{código de conducta} (o \textit{code of conduct}) es una serie de normas que debería seguir la comunidad para evitar ofensas entre colaboradores del repositorio, o a cualquier otro usuario que vaya a usar el programa derivado de este proyecto.
	Para ello usaremos la traducción en español del Contributor Covenant \cite{contributor-covenant}, uno de los dos formatos estándar de este tipo de fichero, usado para proyectos de cualquier tamaño, con una fuerte presencia de la defensa de la equidad, diversidad y justicia.
	\item Un \textbf{archivo de contribución} (o \textit{contributing} \cite{contributing}) es un fichero que explica a quienes quieran colaborar en el repositorio cómo hacerlo.
	
	Esto incluye:
	\begin{itemize}
		\item Tecnologías que se han usado
		\item Estándares de código, y herramientas que faciliten la comprobación de su calidad.
		\item Estándares y procedimientos para informar los cambios y problemas surgidos en la ejecución del proyecto, como los commits, las incidencias y los cambios.
	\end{itemize}
\end{itemize}

Junto a ello, se ha esclarecido el repositorio en busca de archivos que no deberían estar en el repositorio, y se han etiquetado las carpetas para que se pueda entender fácilmente qué se puede encontrar en el directorio.

\begin{table}[H]
	\centering
	\begin{tabularx}{\textwidth}{|>{\columncolor{mintgreen}}c>{\columncolor{mintgreen}}X|}
		\hline
		\includegraphics[width=30pt]{imagenes/Tarea_completada.png} & Con ello, cumplimos los Objetivos \textbf{O-IA 7.} (Leer sobre aquellos estándares y convenciones para facilitar la compartición y liberación del proyecto (por ejemplo. Códigos de conducta, Guías de contribución)) y \textbf{O-DD 6.} (Redactar documentación referida a la arquitectura del proyecto para su posterior mantenimiento y escalabilidad)) \\
		\hline
	\end{tabularx}
\end{table}

\subsection{Aplicación de la licencia al proyecto}
Ya llegados a la recta final de la implementación, se debería tener en cuenta qué licencia podríamos insertar en el Asistente. Para que sea lo más compatible posible con nuestras dependencias, vamos a listar estas dependencias y las licencias a las que están ceñidas:

\begin{center}
	\begin{table}[H]
		\centering
		\begin{tabularx}{\textwidth}{|l|X|X|}
			\hline
			{\cellcolor{lightblue}}\textbf{Dependencia} & {\cellcolor{lightblue}}\textbf{Descripción} & {\cellcolor{lightblue}}\textbf{Licencia} \\
			\hline
			VOSK & Reconocedor de Voz & Apache License 2.0 \\ \hline
			NanoTTS & Sintetizador de Voz & Apache License 2.0  \\ \hline
			PyAudio & Biblioteca de grabación y reproducción de audio & MIT License \\ \hline
			FFMpeg & Biblioteca de grabación y reproducción de Audio & GNU Lesser General Public License (LGPL) version 2.1 or later \\  \hline
			Rasa & Chatbot & Apache License 2.0 \\ \hline
		\end{tabularx}
		\caption{Tabla de licencias que aplican en los proyectos que dependen de Arcadia.}
	\end{table}
\end{center}

Con ello, podemos usar la matriz de compatibilidades de Licencias de Software Libre \cite{matriz-licencias}, donde tendremos en cuenta qué licencia puede compatibilizar LGPL, MIT y Apache (si se puede, claro está). Para ello descargaremos el archivo de hoja de cálculo y buscaremos una licencia cuya columna abarque un Sí en las tres licencias que la componen.

\begin{center}
	\begin{figure}[h]
		\includegraphics[width=\textwidth]{imagenes/MatrizCompatibilidadDinamica.png}
		\includegraphics[width=\textwidth]{imagenes/MatrizCompatibilidadEstatica.png}
		\caption{Matriz de compatibilidad de licencias de Software Libre con respecto a licencias Apache 2.0 (ASL2), LGPL 2.1 o superior (LGPL2.1+), y MIT. Matriz extraída de \cite{matriz-licencias}.}
		\label{fig:matrices-comp}
	\end{figure}
\end{center}

A raíz de observar las matrices en la figura \ref{fig:matrices-comp} , tanto la dinámica como la estática, nos quedaríamos con el siguiente listado de candidatos, los cuales listamos anteriormente en la sección \ref{gpl-license-types}:
\begin{itemize}
	\item Lesser GNU Public License 2.1 (con posibilidad de permitir actualizarse a versiones posteriores)
	\item Lesser GNU Public License 3 (con posibilidad de permitir actualizarse a versiones posteriores)
	\item European Union Public License
	\item GNU Public License 3
	\item Affero GNU Public License 3
\end{itemize}

Teniendo en cuenta que las posibilidades de Arcadia se pueden extender a la red gracias a las acciones de Rasa y su posible extensión a un servicio cliente-servidor donde la fuente de conocimiento residiera en un ordenador remoto y el resto de la implementación se quedaría del lado de nuestro ordenador, se opta finalmente por aplicar una licencia AGPL3, la cual se añade al repositorio junto al resto de la documentación.

\begin{table}[H]
	\centering
	\begin{tabularx}{\textwidth}{|>{\columncolor{mintgreen}}c>{\columncolor{mintgreen}}X|}
		\hline
		\includegraphics[width=30pt]{imagenes/Tarea_completada.png} & Con ello, cumplimos los Objetivos \textbf{O-IA 6.} (Analizar las distintas licencias que se ofrecen y las compatibilidades entre éstas para elegir la idónea para la aplicación resultante.) \\
		\hline
	\end{tabularx}
\end{table}

\subsection{Otros desarrollos que no se han podido completar}
Aparte de lo que se ha realizado para resolver los hitos del proyecto, se han intentado hacer algunas mejoras para facilitar el desarrollo o introducir nuevas funcionalidades, pero han quedado sin éxito tras buscar e investigar sin pdoer conseguir hacer arreglos. Pasamos a listar algunos de ellos:
\begin{itemize}
	\item \textbf{Docker para desarrollo:} Se ha intentado desarrollar un sistema de dos contenedores que hagan ejecutar una instancia de Rasa y otra de nuestro asistente, conectados por una red entre ambos contenedores, con el objetivo de simplificar la ejecución. Al intentar realizarlo, nos hemos encontrado con que al usar las imágenes oficiales del sistema de Chatbot, nos da un error de permisos que no tiene documentada la solución. Tras intentar hacer cambios para que lo aceptara sin éxito, se ha optado por desistir.
	\item \textbf{Segundo hilo para algunas ejecuciones:} Hay cuestiones que precisarían una tarea que se ejecute a la vez que nuestro hilo principal, como puede ser escuchar la radio. Sin embargo, en este momento no se ha conseguido ejecutar un nuevo hilo para estas cuestiones. Se tratará de realizar en un posterior trabajo, permitiendo así la concurrencia y/o paralelismo de las tareas que influyen en Arcadia.
\end{itemize}



	% Presupuesto

	% Conclusiones
	\newpage 
	\chapter{Conclusiones y trabajos futuros}
\section{El proceso de desarrollo en retrospectiva}
\subsection{Lo que ha funcionado}
\subsection{Lo que ha funcionado... más o menos}
\subsection{Lo que no ha funcionado}
\subsection{Tiempo estimado vs. Tiempo real}

\section{¿Y ahora qué? Posibles trabajos y mejoras en el futuro.}

\section{Arcadia, los Asistentes Virtuales y el impacto en la sociedad}

\section{El proceso de aprendizaje y la distribución del conocimiento}

\section{Unas últimas palabras}


	% Trabajos futuros


	
	\newpage
	\bibliography{bibliografia}
	\bibliographystyle{plain}
	
\end{document}

